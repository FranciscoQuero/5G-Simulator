\documentclass[a4paper,11pt]{book}
%\documentclass[a4paper,twoside,11pt,titlepage]{book}
\usepackage{listings}
\usepackage[utf8]{inputenc}
\usepackage[spanish]{babel}
\usepackage{xspace}
\usepackage{multirow}
\usepackage{graphicx}
\usepackage{booktabs}
\usepackage{array}
\usepackage{caption} 
\captionsetup[table]{skip=7.5pt}
% \usepackage[style=list, number=none]{glossary} %
%\usepackage{titlesec}
%\usepackage{pailatino}

\decimalpoint
\usepackage{dcolumn}
\newcolumntype{.}{D{.}{\esperiod}{-1}}
\makeatletter
\addto\shorthandsspanish{\let\esperiod\es@period@code}
\makeatother


%\usepackage[chapter]{algorithm}
\RequirePackage{verbatim}
%\RequirePackage[Glenn]{fncychap}
\usepackage{fancyhdr}
\usepackage{graphicx}
\usepackage{afterpage}

\usepackage{longtable}

\usepackage[pdfborder={0 0 0}]{hyperref} %referencia

% ********************************************************************
% Re-usable information
% ********************************************************************
\newcommand{\myTitle}{Simulador de comunicaciones móviles para redes 5G\xspace}
\newcommand{\myDegree}{Grado en Ingeniería de Tecnologías de Telecomunicación\xspace}
\newcommand{\myName}{Francisco Jesús Quero de la Rosa\xspace}
\newcommand{\myProf}{Pablo Muñoz Luengo\xspace}
\newcommand{\myOtherProf}{Juan Francisco Valenzuela Valdés\xspace}
%\newcommand{\mySupervisor}{Put name here\xspace}
\newcommand{\myFaculty}{Escuela Técnica Superior de Ingenierías Informática y de
Telecomunicación\xspace}
\newcommand{\myFacultyShort}{E.T.S. de Ingenierías Informática y de
Telecomunicación\xspace}
\newcommand{\myDepartment}{Departamento de Teoría de la Señal, Telemática y Comunicaciones\xspace}
\newcommand{\myUni}{\protect{Universidad de Granada}\xspace}
\newcommand{\myLocation}{Granada\xspace}
\newcommand{\myTime}{\today\xspace}
\newcommand{\myVersion}{Version 0.1\xspace}


\hypersetup{
pdfauthor = {Francisco Jesús Quero de la Rosa (fjqr@correo.ugr.es)},
pdftitle = {Simulador de comunicaciones móviles para redes 5G},
pdfkeywords = {Quadriga, 5G}
}

%\hyphenation{}


%\usepackage{doxygen/doxygen}
%\usepackage{pdfpages}
\usepackage{url}
\usepackage{colortbl,longtable}
\usepackage[stable]{footmisc}
%\usepackage{index}

\makeindex
%\usepackage[style=long, cols=2,border=plain,toc=true,number=none]{glossary}
%\makeglossary
\usepackage{acro}
%\usepackage[acronym]{glossaries}
%\makeglossaries
%\DeclareAcronym{1g}{
  short = 1G,
  long  = First Generation,
}

\DeclareAcronym{4g}{
  short = 4G ,
  long  = Fourth Generation,
}

\DeclareAcronym{5g}{
  short = 5G ,
  long  = Fifth Generation,
}

\DeclareAcronym{qos}{
  short = QoS ,
  long  = Quality of Service,
}

\DeclareAcronym{iot}{
  short = IoT ,
  long  = Internet of Things,
}

\DeclareAcronym{tic}{
  short = TIC ,
  long  = Tecnologías de la Información y Comunicación,
}
\DeclareAcronym{hetnet}{
  short = HetNet ,
  long  = Heterogeneous Network,
}
\DeclareAcronym{mimo}{
  short = MIMO ,
  long  = Multiple Input Multiple Output,
}
\DeclareAcronym{icic}{
  short = ICIC ,
  long  = Inter-cell Interference Coordination,
}
\DeclareAcronym{bs}{
	short = BS,
    long = Base Station,
}
\DeclareAcronym{los}{
	short = LOS,
    long = Line-of-Sight,
}

\DeclareAcronym{nlos}{
	short = NLOS,
    long = Non-Line-of-Sight,
}

\DeclareAcronym{o2i}{
	short = O2I,
    long = Outdoor-To-Indoor,
}

\DeclareAcronym{wise}{
	short = WiSE,
    long = Wireless Simulator Evolution,
}

\DeclareAcronym{nyusim}{
	short = MYUSIM,
    long = New York University SIMulator,
}

\DeclareAcronym{ofdm}{
	short = OFDM,
    long = Orthogonal Frequency-Division Multiplexing,
}

\DeclareAcronym{fbmc}{
	short = FBMC,
    long = Filter Bank Multi-Carrier,
}

\DeclareAcronym{lte}{
	short = LTE,
    long = Long-Term Evolution,
}

\DeclareAcronym{sinr}{
    short = SINR,
    long = Signal-to-Interference-plus-Noise Ratio,
}

\DeclareAcronym{ca}{
    short = CA,
    long = Carrier Agregation,
}

\DeclareAcronym{gui}{
    short = GUI,
    long = Graphic User Interface,
}

\DeclareAcronym{tfg}{
    short = TFG,
    long = Trabajo de Fin de Grado,
}

\DeclareAcronym{lgpl}{
    short = LGPL,
    long = Lesser General Public License,
}

\acsetup{first-long-format=\itshape}


% Definición de comandos que me son tiles:
\renewcommand{\indexname}{Índice alfabético}
%\renewcommand{\acroname}{Glosario}

\pagestyle{fancy}
\fancyhf{}
\fancyhead[LO]{\leftmark}
\fancyhead[RE]{\rightmark}
\fancyhead[RO,LE]{\textbf{\thepage}}
\renewcommand{\chaptermark}[1]{\markboth{\textbf{#1}}{}}
\renewcommand{\sectionmark}[1]{\markright{\textbf{\thesection. #1}}}

\setlength{\headheight}{1.5\headheight}

\newcommand{\HRule}{\rule{\linewidth}{0.5mm}}
%Definimos los tipos teorema, ejemplo y definición podremos usar estos tipos
%simplemente poniendo \begin{teorema} \end{teorema} ...
\newtheorem{teorema}{Teorema}[chapter]
\newtheorem{ejemplo}{Ejemplo}[chapter]
\newtheorem{definicion}{Definición}[chapter]

\definecolor{gray97}{gray}{.97}
\definecolor{gray75}{gray}{.75}
\definecolor{gray45}{gray}{.45}
\definecolor{gray30}{gray}{.94}

\lstset{ frame=Ltb,
     framerule=0.5pt,
     aboveskip=0.5cm,
     framextopmargin=3pt,
     framexbottommargin=3pt,
     framexleftmargin=0.1cm,
     framesep=0pt,
     rulesep=.4pt,
     backgroundcolor=\color{gray97},
     rulesepcolor=\color{black},
     %
     stringstyle=\ttfamily,
     showstringspaces = false,
     basicstyle=\scriptsize\ttfamily,
     commentstyle=\color{gray45},
     keywordstyle=\bfseries,
     %
     numbers=left,
     numbersep=6pt,
     numberstyle=\tiny,
     numberfirstline = false,
     breaklines=true,
   }
 
% minimizar fragmentado de listados
\lstnewenvironment{listing}[1][]
   {\lstset{#1}\pagebreak[0]}{\pagebreak[0]}

\lstdefinestyle{CodigoC}
   {
	basicstyle=\scriptsize,
	frame=single,
	language=C,
	numbers=left
   }
\lstdefinestyle{CodigoC++}
   {
	basicstyle=\small,
	frame=single,
	backgroundcolor=\color{gray30},
	language=C++,
	numbers=left
   }

 
\lstdefinestyle{Consola}
   {basicstyle=\scriptsize\bf\ttfamily,
    backgroundcolor=\color{gray30},
    frame=single,
    numbers=none
   }


\newcommand{\bigrule}{\titlerule[0.5mm]}


%Para conseguir que en las páginas en blanco no ponga cabeceras
\makeatletter
\def\clearpage{%
  \ifvmode
    \ifnum \@dbltopnum =\m@ne
      \ifdim \pagetotal <\topskip
        \hbox{}
      \fi
    \fi
  \fi
  \newpage
  \thispagestyle{empty}
  \write\m@ne{}
  \vbox{}
  \penalty -\@Mi
}
\makeatother

\usepackage{pdfpages}
\DeclareAcronym{1g}{
  short = 1G,
  long  = First Generation,
}

\DeclareAcronym{4g}{
  short = 4G ,
  long  = Fourth Generation,
}

\DeclareAcronym{5g}{
  short = 5G ,
  long  = Fifth Generation,
}

\DeclareAcronym{qos}{
  short = QoS ,
  long  = Quality of Service,
}

\DeclareAcronym{iot}{
  short = IoT ,
  long  = Internet of Things,
}

\DeclareAcronym{tic}{
  short = TIC ,
  long  = Tecnologías de la Información y Comunicación,
}
\DeclareAcronym{hetnet}{
  short = HetNet ,
  long  = Heterogeneous Network,
}
\DeclareAcronym{mimo}{
  short = MIMO ,
  long  = Multiple Input Multiple Output,
}
\DeclareAcronym{icic}{
  short = ICIC ,
  long  = Inter-cell Interference Coordination,
}
\DeclareAcronym{bs}{
	short = BS,
    long = Base Station,
}
\DeclareAcronym{los}{
	short = LOS,
    long = Line-of-Sight,
}

\DeclareAcronym{nlos}{
	short = NLOS,
    long = Non-Line-of-Sight,
}

\DeclareAcronym{o2i}{
	short = O2I,
    long = Outdoor-To-Indoor,
}

\DeclareAcronym{wise}{
	short = WiSE,
    long = Wireless Simulator Evolution,
}

\DeclareAcronym{nyusim}{
	short = MYUSIM,
    long = New York University SIMulator,
}

\DeclareAcronym{ofdm}{
	short = OFDM,
    long = Orthogonal Frequency-Division Multiplexing,
}

\DeclareAcronym{fbmc}{
	short = FBMC,
    long = Filter Bank Multi-Carrier,
}

\DeclareAcronym{lte}{
	short = LTE,
    long = Long-Term Evolution,
}

\DeclareAcronym{sinr}{
    short = SINR,
    long = Signal-to-Interference-plus-Noise Ratio,
}

\DeclareAcronym{ca}{
    short = CA,
    long = Carrier Agregation,
}

\DeclareAcronym{gui}{
    short = GUI,
    long = Graphic User Interface,
}

\DeclareAcronym{tfg}{
    short = TFG,
    long = Trabajo de Fin de Grado,
}

\DeclareAcronym{lgpl}{
    short = LGPL,
    long = Lesser General Public License,
}

\acsetup{first-long-format=\itshape}


\begin{document}
\begin{titlepage}
 
 
\newlength{\centeroffset}
\setlength{\centeroffset}{-0.5\oddsidemargin}
\addtolength{\centeroffset}{0.5\evensidemargin}
\thispagestyle{empty}

\noindent\hspace*{\centeroffset}\begin{minipage}{\textwidth}

\pagenumbering{arabic} % para empezar la numeración con números árabes

\centering
\includegraphics[width=0.9\textwidth]{imagenes/logo_ugr.jpg}\\[1.4cm]

\textsc{ \Large TRABAJO FIN DE GRADO\\[0.2cm]}
\textsc{INGENIERÍA DE TECNOLOGÍAS DE TELECOMUNICACIÓN}\\[1cm]
% Upper part of the page
% 
% Title
{\huge\bfseries \myTitle\\
}
\noindent\rule[-1ex]{\textwidth}{3pt}\\[3.5ex]
{\large\bfseries Integración del simulador de canal de propagación \textit{QuaDRiGa} en un simulador de nivel de sistema}
%\end{minipage}

%\vspace{2.5cm}
%\noindent\hspace*{\centeroffset}\begin{minipage}{\textwidth}
\centering

\textbf{Autor}\\ {\myName{}}\\[2.5ex]
\textbf{Directores}\\
{\myProf\\
\myOtherProf}\\[2cm]
\includegraphics[width=0.3\textwidth]{imagenes/etsiit_logo.png}\\[0.1cm]
\textsc{Escuela Técnica Superior de Ingenierías Informática y de Telecomunicación}\\
\textsc{---}\\
\myTime
\end{minipage}
%\addtolength{\textwidth}{\centeroffset}
%\vspace{\stretch{2}}
\end{titlepage}



%\chapter*{}
%\thispagestyle{empty}
%\cleardoublepage

%\thispagestyle{empty}

\input{portada/portada_2}

\cleardoublepage
\thispagestyle{empty}

\begin{center}
{\large\bfseries \myTitle: Integración del simulador de canal de propagación \textit{QuaDRiGa} en un simulador de nivel de sistema}\\
\end{center}
\begin{center}
\myName\\
\end{center}

%\vspace{0.7cm}
\noindent{\textbf{Palabras clave}: 5G, Simulador, Redes 5G, HetNets, Comunicaciones de Radio, Comunicaciones Inalámbricas, IoT, Macro-Celdas, Micro-Celdas, QuaDRiGa, Capacidad de Canal, Simulador de Nivel de Sistema }\\

\vspace{0.7cm}
\noindent{\textbf{Resumen}}\\

La evolución de las tecnologías inalámbricas está marcada por el gran crecimiento de usuarios y de terminales móviles que hacen uso de las infraestructuras de comunicación. Tal es el cambio que las telecomunicaciones inalámbricas están sufriendo que se espera que en los próximos tres años, el número de usuarios de las infraestructuras se vea incrementado en más de un 15\% a nivel mundial, con un volumen de tráfico de red tres veces mayor que el actual.

Con la finalidad de satisfacer los rigurosos requisitos de las nuevas tecnologías de radiocomunicación y facilitar su planificación y despliegue, han sido impulsadas ciertas iniciativas por parte del sector industrial y el sector académico. Una de ellas es \textit{QuaDRiGa}, una herramienta de simulación que ha sido utilizada por diversos proyectos europeos para generar respuestas de canales de radio realistas que son compatibles con los modelos estandarizados por el 3rd Generation Partnership Project, 3GPP. 

Este proyecto amplía las funcionalidades de \textit{QuaDRiGa} para producir simulaciones a nivel de red de escenarios 5G, con la adición de cierta cantidad de mejoras que hacen que las simulaciones otorguen resultados significativos, de una forma eficiente en cuanto a los aspectos computacionales se refiere. Este simulador, denominando \textit{5Gneralife} ha sido desarrollado en Matlab con el ánimo de que resulte fácil de usar y que contribuya a los fines académicos.

Además, se realiza una serie de evaluaciones del simulador en variedad de escenarios, desde escenarios realistas para estudiar el rendimiento de la red, hasta modificando parámetros con el propósito de determinar cómo afectan las configuraciones de la infraestructura al desempeño de la misma en términos de capacidad y cobertura.
\cleardoublepage


\thispagestyle{empty}


\begin{center}
{\large\bfseries \textit{5Gneralife}: Mobile communications simulator for 5G networks}\\
\end{center}
\begin{center}
Francisco Quero de la Rosa\\
\end{center}

%\vspace{0.7cm}
\noindent{\textbf{Keywords}: 5G, Simulator, HetNets, Radio-Communications, Wireless Communications, IoT, Large Cells, Small Cells, QuaDRiGa, Channel Capacity, Network Level Simulator}\\

\vspace{0.7cm}
\noindent{\textbf{Abstract}}\\

Wireless technologies evolution is influenced by the strong user and mobile terminals growth. There is such a change of wireless telecommunication paradigm that users amount around world is being expected to grow by 15 percent in the next three years, with a traffic volume boost from 17 to 50 exabytes.

With the aim of satisfying the stringent requirements of the new emerging technologies and facilitating their planning and deployment, numerous initiatives have been taken from the industry and academia sectors. One of them is \textit{QuaDRiGa}, a simulation-tool currently used in several European projects to generate realistic radio channel impulse responses which are compatible with \textit{3rd Generation Partnership Project, 3GPP} standardized channel models. 

This project extends \textit{QuaDRiGa} to produce network-level simulations of relevant 5G scenarios in a computationally-efficient manner, which includes several improvements which contribute to achieve a more versatile simulator. This 5G simulator, called \textit{5Gneralife}, has been developed in Matlab with the aim of being intuitive in order to hope to be useful to the academic community.

In addition, performance evaluations in terms of capacity and coverage are provided for several realistic use cases, as well as some experimentation simulations that have as purpose to determine how the enviroment is affected by parameter variation.

\chapter*{}
\thispagestyle{empty}

\noindent\rule[-1ex]{\textwidth}{2pt}\\[4.5ex]

Yo, \textbf{\myName}, alumno de la titulación \myDegree de la \textbf{Escuela Técnica Superior
de Ingenierías Informática y de Telecomunicación de la Universidad de Granada}, con DNI 77382810-T, autorizo la
ubicación de la siguiente copia de mi Trabajo Fin de Grado en la biblioteca del centro para que pueda ser
consultada por las personas que lo deseen.

\vspace{6cm}

\noindent Fdo: \myName

\vspace{2cm}

\begin{flushright}
Granada a \myTime.
\end{flushright}


\chapter*{}
\thispagestyle{empty}

\noindent\rule[-1ex]{\textwidth}{2pt}\\[4.5ex]

D. \textbf{\myProf}, Profesor del Área de Telemática del Departamento \myDepartment de la Universidad de Granada.

\vspace{0.5cm}

D. \textbf{\myOtherProf}, Profesor del Área de Telemática del Departamento \myDepartment de la Universidad de Granada.


\vspace{0.5cm}

\textbf{Informan:}

\vspace{0.5cm}

Que el presente trabajo, titulado \textit{\textbf{\myTitle}},
ha sido realizado bajo su supervisión por \textbf{\myName}, y autorizamos la defensa de dicho trabajo ante el tribunal
que corresponda.

\vspace{0.5cm}

Y para que conste, expiden y firman el presente informe en Granada a \myTime.

\vspace{1cm}

\textbf{Los directores:}

\vspace{5cm}

\noindent \textbf{\myProf} \ \ \ \ \ \textbf{\myOtherProf}

\chapter*{Agradecimientos}
\thispagestyle{empty}

       \vspace{1cm}


Quiero dedicar unas palabras a todas las personas que han formado parte de mis últimos cuatro años de andanzas, sin ellos, todo habría sido más cuesta arriba.

En especial, a mis padres, Francisco y Ana Mari,los mejores que se pueden tener, que me han dado todas las facilidades del mundo para elegir la senda que yo quería, sin importar nada más que ofrecerme lo mejor para mí. Y aún a día de hoy, con todo acabado, siguen ofreciéndomelo. Y a mi familia que tantos ratos de charla sobre cómo me iba, y que tantas palabras de ánimo me han regalado.

Cómo no, también a Ana por todo su cariño y compañía en los días que más lo he necesitado (y por soportarme), por haber querido compartir conmigo el final de esta etapa, y por querer compartir el comienzo de la siguiente.

A mis tutores, Pablo y Juanfra por su total disponibilidad siempre que lo he necesitado, y también por su paciencia y magníficas labores docentes y de tutelaje no solo académico, sino también personal y profesional. 

A ese porcentaje reducido de profesores y profesoras que transmiten más que conocimientos, que han despertado en mí inquietudes que no existían y que se pueden considerar verdaderos docentes.

A mis compañeros del Grado, que tantas horas hemos pasado juntos, en especial a Carmelo, Sergio, David, Adri, Gabri y Gásquez, cuya complicidad demuestra que el grado me ha regalado más que compañeros, me ha regalado amigos inseparables.

A mis amigos del CITIC, Jose Manuel, Pilar, José Antonio, Óscar, Ángel y Pablo, con los que he compartido mis últimos seis meses del grado entre risas, café y trabajo duro.

A todos mis amigos de Porcuna que, por haber elegido Granada como lugar de estudio, con todo mi pesar he tenido que dejar de ver tanto, y que aun así, siempre están cuando se les necesita.

Y en definitiva, a toda persona que haya vivido conmigo parte de esta carrera de fondo que no acaba más que empezar. Aunque me haya costado algún sacrificio, hoy, más que nunca, me alegro de haber tomado este camino. Sé que me he extendido pero todo agradecimiento se queda corto para aquellos que lo merecen. Gracias.
%\frontmatter
\tableofcontents
\listoffigures
\listoftables
\clearpage
\printacronyms[name={Acrónimos}]
%
%\mainmatter
%\setlength{\parskip}{5pt}

\chapter{Introducción}\label{cap.introduccion}
%\pagenumbering{arabic} % para empezar la numeración con números
\section{Contexto}
Desde el surgimiento de la primera generación de comunicaciones móviles \ac{1g}, la impronta de estas comunicaciones se ve reflejada hasta en el más mínimo aspecto de la vida cotidiana y de la industria. Lejos ha quedado ya la tradicional y exclusiva utilidad de realizar llamadas de voz que tenían los terminales móviles en un principio.

\begin{figure}[hb]
	\centering
    \includegraphics[width=1\linewidth]{imagenes/5gforecast.PNG}
	\caption{Suscripciones móviles por tecnología (en miles de millones) \cite{5gForecastEricsson}}
	\label{fig:5gforecast}
\end{figure}

Con la aparición de las siguientes generaciones, el uso de datos móviles se ha convertido en la finalidad principal de estos dispositivos. Según \cite{informeInicial}, ``El móvil es el dispositivo más utilizado en España para acceder a internet, usado ya por el 94,6 \% de los españoles", asimismo, según la previsión de Cisco \cite{informeInicialCisco}, para el año 2021 existirá un total de 11,6 miles de millones de dispositivos, tanto terminales móviles como otros tipos de dispositivos, conectados a la red móvil en todo el mundo. Esto supondría un tráfico mensual de alrededor de 50 Exabytes a través de la infraestructura de comunicaciones móviles -tres veces más tráfico que en la actualidad-.

\subsection{El futuro de las redes inalámbricas. 5G, \textit{hetnets} y Horizonte 2020}

Con el propósito de hacer frente a las características que las infraestructuras de red han de reunir en el futuro, resultan cruciales las labores de investigación y desarrollo para el establecimiento de nuevos estándares que permitan la coexistencia de dispositivos de distinta naturaleza, ya que las actuales tendencias hacen que empiece a surgir la distinción entre Internet de terminales móviles e Internet de las Cosas, \ac{iot}, siendo de esperar que el número de dispositivos conectados crezca considerablemente año tras año.


Actualmente, es la comunicación de quinta generación, \ac{5g} el estándar que se encuentra en pleno desarrollo y que será el sucesor de la cuarta generación, \ac{4g}. Este estándar pretende ofrecer servicios con una muy alta capacidad, conectividad masiva, muy baja latencia, muy alta seguridad, consumo de energía muy bajo y una calidad de servicio, \ac{qos} extremadamente alta \cite{comparative5G}.

\begin{table}[h]
\centering
\caption{Objetivos de 5G \cite{cognitive5G}.}
\label{tab:5gfeatures}
\resizebox{\textwidth}{!}{%
\begin{tabular}{@{} >{\centering\arraybackslash}m{2.5cm} >{\centering\arraybackslash}m{3.5cm} m{7cm} @{}}
\toprule
\multicolumn{2}{c}{\textbf{Escenario de Aplicación}}    & \multicolumn{1}{c}{\textbf{Requisitos y Desafíos}}                                                                                                         \\ \midrule
\multirow{2}{*}[-17pt]{Internet móvil} & Cobertura extensa & Ofrecer un servicio de alta velocidad, en cualquier momento, en cualquier sitio y en escenarios difíciles como áreas remotas.                              \\ \cmidrule(l){2-3} 
                                & Capacidad masiva      & Ofrecer servicio a usuarios con ratios de transmisión extremadamente altos, y hallar las características que las redes de alto flujo de datos necesitarán. \\ \midrule
\multirow{2}{*}[-30pt]{IoT}            & Conectividad masiva   & Ofrecer capacidad para más de un millón de conexiones simultáneas, y asegurar a los terminales un consumo extremadamente bajo de energía.                  \\ \cmidrule(l){2-3} 
                                & Baja Latencia         & Ofrecer a los usuarios un retardo de menos de 1 milisegundo punto a punto, y cerca de un 100\% de fiabilidad.                                                       \\ \bottomrule
\end{tabular}%
}
\end{table}

En concreto, 5G propone un enfoque disruptivo de las comunicaciones. Debido a la gran cantidad de terminales que estarán conectados simultáneamente y su constante expansión, es primordial ofrecer servicio a todos ellos, tanto en entornos con una densidad de usuarios extremadamente alta como en entornos rurales en los que existe una mayor dispersión de usuarios. Para ello, los operadores pretenden ampliar el rendimiento de sus servicios a través de sistemas multiantena utilizando tecnología \ac{mimo} y mediante técnicas de modulación y multiplexación más eficientes, al mismo tiempo que se aumenta el ancho de banda y se utilizan frecuencias mucho más altas.


\begin{figure}[ht]
	\centering
    \includegraphics[width=1\linewidth]{imagenes/hetnet_enviroment.png}
	\caption{Entorno típico de \acs{hetnet} \cite{hetnetexplained}}
	\label{fig:hetnet}
\end{figure}

Sin embargo, este conjunto de técnicas y adaptaciones no son suficientes para soportar el uso extremadamente intenso que recibirán las infraestructuras. Por ello, un concepto muy importante en 5G es el de \textit{\acl{hetnet}} (\acs{hetnet}), o lo que es lo mismo, redes heterogéneas en las que conviven nodos o puntos de acceso de diferentes características. En las redes heterogéneas las celdas cuentan con un diferente tamaño según su tipo. De este modo, se distingue entre \textit{macro-, micro-, pico-} y \textit{femto-celdas}. La única diferencia entre ellas a priori es el área de cobertura, la cual es definida mediante la variación de la frecuencia, de la potencia de transmisión y de la altura a la que se encuentra la estación base de la celda en cuestión.


\begin{table}[ht]
\centering
\caption{Características de las clases de celdas.}
\label{tab:celdas}
\begin{tabular}{m{3cm} m{9cm}}
\hline
\multicolumn{1}{c}{\textbf{Tipo de celda}} & \multicolumn{1}{c}{\textbf{Características}}                                                                                                             \\ \hline
Femto-celda                                & Son autónomas y tienen capacidad para apenas unos pocos usuarios. El área que puede cubrir es muy reducida.                                              \\ \hline
Pico-celda                                 & Pueden soportar hasta 100 usuarios simultáneos en áreas de menos de 200 metros. Normalmente se usan en interiores.                                       \\ \hline
Micro-celda                                & De uso urbano, estas celdas pueden cubrir áreas de hasta 1,5 km. En la actualidad, se utilizan para proporcionar cobertura adicional en eventos masivos. \\ \hline
Macro-celda                                & Se puede comparar con las celdas tradicionales. Ofrecen una cobertura máxima de unos 30 km.                                                              \\ \hline
\end{tabular}
\end{table}


A modo de simplificación, se suele utilizar el término \textit{small cell} para hacer referencia a cualquier celda que no sea del tipo \textit{macro-celda}. Se puede comprender un escenario típico de red heterogénea a través de la ilustración de la figura \ref{fig:hetnet}. Como explica dicha figura, en el paradigma de las \acs{hetnet}, las \textit{small cells} se utilizan con triple finalidad: ofrecer cobertura en zonas que las celdas convencionales no cubren, reducir la carga de las macro-celdas y proporcionar servicio en zonas con mayor demanda como interiores o puntos de alta densidad de usuarios (\textit{hot-spots}). Además, 5G pretende incorporar nuevos usos gracias al hecho de disponer de varias estaciones base distintas conviviendo a la misma vez \cite{hetnetexplained}. 

Por ejemplo, una de las posibles mejoras que las redes heterogéneas ofrecen es la de \acl{icic}, ICIC, la cual consiste en la utilización de la interfaz extra con la que cuentan las BS para comunicarse entre ellas con el fin  de reducir interferencias. Para ilustrar el potencial que este conjunto de técnicas ofrece, una de sus implementaciones consiste en delegar el enlace de subida y el enlace de bajada a distintas estaciones base cada uno, de modo que si existe interferencia en uno de los enlaces al nodo conectado, ésta se pueda mitigar utilizando el enlace de menor interferencia para tal uso. Dicha técnica ya ha sido implementada en algunas revisiones de LTE pero gracias a las posibilidades que ofrecen las \textit{small cells}, el rendimiento del enlace podría mejorar notablemente, ya que para 4G solo era posible en casos en los que se recibiera una correcta señal de dos estaciones base a la vez -algo complicado en escenarios con un reuso de frecuencias bajo-, mientras que en redes heterogéneas, son más frecuentes las situaciones cuya calidad de enlace sea aceptable para \textit{small cells} y para \textit{large cells} simultáneamente \cite{ieeeicic}. 

\begin{figure}[!hb]
	\centering
    \includegraphics[width=0.75\linewidth]{imagenes/icic.png}
	\caption{Escenario simple en el que se utiliza una técnica de \acs{icic} para evitar interferencia en el enlace de bajada.}
	\label{fig:icic}
\end{figure}

Como se ha comentado anteriormente y como también se mostraba en el cuadro \ref{tab:5gfeatures}, los retos que se plantean para el desarrollo de 5G no son triviales. Con la finalidad de promover una investigación lo suficientemente eficaz como para abarcar todos los problemas que aparecen sobre la mesa, desde la Administración Pública se han consolidado programas impulsores como el Horizonte 2020 de la Unión Europea, el cual ofrece financiación para proyectos de cualquier naturaleza, incluidos los proyectos de \ac{tic}. Específicamente, el interés en el área \acs{tic} del Horizonte 2020 es conseguir los suficientes avances de consolidación de 5G y de infraestructuras \acs{tic} modernas, sobre todo si explotan el paradigma y/o hacen uso de entornos de IoT, Smart Cities o Big Data, a la vez que se hace hincapié en la ciberseguridad.

Gracias a dicha financiación, pueden ver la luz proyectos como \textit{QuaDRiGa} \cite{quadriga}, un simulador de canal escrito para la plataforma \textit{Matlab} que permite obtener datos realistas de comunicaciones basadas en entornos 5G y \acs{hetnet}. Dicho software será utilizado en este Trabajo de Fin de Grado y, por ello, se dedicará el Capítulo \ref{cap.quadriga} a analizarlo y describirlo con el suficiente detenimiento.

\section{Motivación}

Tal y como se habrá podido deducir de la sección anterior, un reto que suponen las redes heterogéneas es la complejidad de su planificación. El despliegue de numerosas \textit{small cells} junto a las celdas convencionales o nuevas \textit{large cells} agregan un esfuerzo adicional a la hora del diseño de las redes. Es obvia la necesidad de conocer el rendimiento a priori de las redes antes de su desdoble debido a su gran coste y a la suma importancia de ofrecer un servicio óptimo, sin sobredimensionar a la misma vez que se abastece a los usuarios sin sobrecargas.
Además, para un correcto desarrollo de los estándares venideros, como \acs{5g}, es preciso tener en mente los requisitos demandados por el futuro uso de los mismos. No es posible realizar un diseño de un estándar con tanto potencial como 5G sin una reproducción de resultados fiable.

Debido a tal problemática, en etapas de diseño, desarrollo, planificación y despliegue de redes y nuevas tecnologías inalámbricas, los simuladores toman un papel muy importante, puesto que permiten anticiparse a resultados con la finalidad de evitar futuros inconvenientes que un mal diseño pueda provocar, y su consecuente repercusión económica. Del mismo modo, los simuladores también permiten conocer los propios límites de una tecnología y su rendimiento. Algo que aunque solamente tenga un uso inmediato en los campos de investigación, a largo plazo permite evolucionar gracias a revisiones de dichas tecnologías y mejoras en siguientes versiones.

Es por ello que la finalidad del presente Trabajo de Fin de Grado es la elaboración de un simulador de comunicaciones móviles de entornos heterogéneos, con enfoque en \acs{5g}, llamado \textit{5Gneralife} -\textit{5G geNERAtor of LInk level simulations For Educational purposes}-.

\section{Objetivos}

El principal objetivo de este proyecto es utilizar como base el generador de canal de radio \textit{QuaDRiGa} utilizando la plataforma de programación y desarrollo \textit{Matlab} para el desarrollo de un simulador funcional de \acs{5g} que, como elementos fundamentales, incluya terminales móviles y \ac{bs} de tipo micro-celda y macro-celda, a la vez que permita obtener resultados básicos -potencia recibida, interferencias, capacidad de canal...- para distintos criterios de emparejamiento entre estaciones base, y terminales.

Para ello, en primer lugar el alumno debe familiarizarse con el entorno de \textit{QuaDriGa}, teniendo en cuenta que es un simulador de nivel de enlace que está capacitado para su uso a nivel de sistema, así como con el paradigma de comunicaciones móviles, indagando en 5G y en redes heterogéneas especialmente para conocer sus conceptos, metodologías y técnicas.

Seguidamente, es necesario hacer una planificación de las características que el simulador reunirá, descartando las que resulten inviables o aquellas que \textit{QuaDRiGa} no haga posible e implementarlas, modificando el código fuente de \textit{QuaDRiGa} si fuera necesario, y desarrollando sus propios \textit{scripts} y funciones.

El objetivo fundamental de estas etapas es el de extender el simulador de capa física \textit{QuaDRiGa} para obtener un simulador de nivel de enlace más completo con nuevas funcionalidades, incluyendo algunas características propias de simuladores de nivel de sistema, así como también una mayor flexibilidad para evaluar escenarios multi-celda y multi-frecuencia.

Por último, en base al trabajo realizado y al software implementado, se realizarán las oportunas pruebas comprobando el rendimiento de distintos escenarios y ajustes de comunicaciones, así como una especificación de orientaciones de uso para que las simulaciones resulten realistas.

\section{Organización de la memoria}

A modo de antesala del contenido del proyecto, a continuación se detalla un listado de capítulos y un resumen de sus contenidos, con la finalidad de aclarar qué se puede esperar en el proyecto y su organización.

\paragraph{Capítulo 1: Introducción} \mbox{} \\
	Este capítulo está dedicado a introducir la motivación del proyecto, su importancia y sus objetivos, así como realizar una primera toma de contacto con las herramientas que se utilizarán. También se realiza una descripción de la organización de la memoria.

\paragraph{Capítulo 2: Estado del arte} \mbox{} \\
	En él se describen las soluciones que existen en la actualidad en cuanto a simuladores 5G se refiere. Se realiza una breve evaluación de los principales simuladores 5G, independientemente de su clasificación.

\paragraph{Capítulo 3: Planificación y requisitos} \mbox{} \\
	Se establecen unos requisitos que se consideran imprescindibles para que, una vez dada la finalización del proyecto, este resulte satisfactorio. Además, se expone la planificación que se llevó a cabo para la realización del proyecto así como una estimación del coste que conllevaría.

\paragraph{Capítulo 4: Acerca de \textit{QuaDRiGa}} \mbox{} \\
	En este capitulo se lleva a cabo una descripción detallada de \textit{QuaDRiGa}, exponiendo sus funcionalidades, limitaciones, su paradigma de programación y su estructura. Finalmente, se realiza una pequeña prueba de dicho software.

\paragraph{Capítulo 5: Diseño e implementación} \mbox{} \\
    Este capítulo contiene una explicación detallada de toda la funcionalidad del código del proyecto, su diseño y estructura, una descripción de sus características y funcionalidades y, por último, una guía de uso.

\paragraph{Capítulo 6: Pruebas} \mbox{} \\
    Se realiza una serie de pruebas al simulador para comprobar su capacidad y evaluar los resultados obtenidos así como el consumo de recursos. También se extraen observaciones y características de los entornos simulados.

\paragraph{Capítulo 7: Conclusiones} \mbox{} \\
    Conclusiones, defectos y puntos fuertes \textit{5Gneralife}. Alcance del proyecto. Se detallan posibles mejoras y futuros pasos para el proyecto, así como sus perspectivas de futuro.

%\section{Alcance del trabajo}
%
\chapter{Estado del arte}\label{cap.estado del arte}

\section{Introducción}

Puesto que en el anterior capítulo se hizo especial hincapié en la importancia de los simuladores, este segundo capítulo está dedicado a exponer las principales implementaciones de simuladores orientados a 5G que existen en la actualidad, resaltando las diferencias entre ellas. Por último, se justifican las características que, como objetivo, debe reunir el simulador implementado en el actual proyecto.

Antes de describir los diferentes simuladores existentes a día de hoy, es importante tener en cuenta los tipos de nivel funcional de los simuladores de comunicaciones móviles atendiendo a sus características:
\begin{itemize}
    \item \textbf{Simulador de capa física:} modela las comunicaciones básicas entre un transmisor y receptor teniendo en cuenta parámetros como la posición de los mismos, la frecuencia, el tipo de canal, la trayectoria de la onda y las pérdidas de propagación.
    \item \textbf{Simulador de nivel de enlace:} cuenta con la capacidad de calcular parámetros como la cobertura, potencia o análisis de obstáculos. Además incluye funcionalidades -a veces limitadas- de emparejamiento terminal-estación base y/o control de errores.
    \item \textbf{Simulador de nivel de sistema:} computa parámetros y datos específicos como control de admisión, gestión de carga o eficiencia, incluyendo en ocasiones posibilidad de estudio de uso de ciertos protocolos en la red.
\end{itemize}

A continuación, se enumeran los principales simuladores de 5G desarrollados tanto a nivel de enlace como a nivel de sistema.

\section{NYUSIM Channel Simulator}
\subsection{Introducción}
\ac{nyusim} \cite{nyusim} es un simulador de canal (nivel de enlace) escrito en \textit{Matlab} con el ánimo de generar respuestas al impulso de diversos canales, en dominio espacial y en dominio temporal, con compatibilidad con el modelo de 3GPP pero con ciertas diferencias según sus propios desarrolladores explican en sus publicaciones. Ha sido desarrollado por la Escuela de Ingeniería de la Universidad de Nueva York y está disponible para su libre descarga y distribución en su página web.

Se considera el análogo de \textit{QuaDRiGa} debido a que reúnen características similares y la misma concepción: son simuladores de canal que pretenden modelar comunicaciones móviles de acuerdo a los estándares de 3GPP.
\subsection{Características}
La característica que abandera a este simulador es su capacidad de simular enlaces de radio en frecuencias entre 2 y 73 GHz, un rango bastante extenso que tiene en cuenta las frecuencias de todas las futuras comunicaciones 5G. Sus estaciones base cuentan con la posibilidad de ser modeladas mediante antenas MIMO. Además, considera numerosos tipos de escenario, distinguiendo los casos con y sin visión directa, \ac{los}, junto con pérdidas por penetración de exterior a interior, \ac{o2i}, y el modelado clásico de 3GPP para casos sin visión directa, \ac{nlos}, de descomponer la señal en un total de 20 trayectorias distintas, cada uno con un retardo, una amplitud y un desfase distintos.

\section{WiSE}
\subsection{Introducción}
\ac{wise} \cite{wise} es un simulador de nivel de sistema para redes móviles \acs{lte} y 5G concebido en un principio para su uso exclusivo en simulaciones 4G. Fue desarrollado por investigadores de la \textit{National Central University} de Taiwan y ha sido validado en campañas de calibración del 3GPP.
\subsection{Características}
\acs{wise} integra la capacidad de simular celdas de entornos urbano y rural, tanto micro celdas como macro-celdas así como \textit{Hot-spots} de interiores y alta densidad. Soporta frecuencias de hasta 100 GHz y modelado de canal en 3D. Además, integra posibilidad de movilidad de usuarios, de simular edificios y de grandes anchos de banda.

\section{GTEC}
\subsection{Introducción}
Desarrollado por el Grupo de Tecnología Electrónica y Comunicaciones de la Universidad de A Coruña, GTEC \cite{gtec} se trata de un simulador de nivel de enlace modular que integra la evaluación de entornos de 4G y 5G. Aunque no está basado en simulaciones estándares como los anteriores simuladores, ofrece la posibilidad de ser personalizado y configurado al gusto, gracias a su estructura modular que permite modificar cualquier parámetro fácilmente.
\subsection{Características}
Además de su peculiaridad de ser modular, el simulador GTEC cuenta con capacidad de crear capas de simulación de enlaces que utilizan \ac{ofdm} o bien \ac{fbmc} como forma de onda. Además, cuenta con diferentes modelos de canal obtenidos de resultados experimentales propios y permite realizar medidas al momento en diferentes escenarios.

\section{Vienna}
\subsection{Introducción}
Vienna \cite{vienna} es una \textit{suite} de simuladores escritos en \textit{Matlab} que incluye simuladores de nivel de enlace y de nivel de sistema para \acs{lte} y actualmente, solo un simulador a nivel de enlace para 5G, mientras que el simulador de nivel de sistema se encuentra en desarrollo, todos ellos implementados por investigadores de la Universidad Técnica de Viena. Se distribuye con una licencia de uso exclusivamente académico.
\subsection{Características}
El simulador de nivel de enlace de 5G actualmente incluye características distintivas con respecto a los demás simuladores como la elección de parámetros individualmente para cada nodo, la separación de los canales de subida y de bajada, simulación multi-portadora o diversidad de receptores. Sin embargo, no admite planificación geométrica de la red ni cuenta con un modelo de pérdidas de propagación.

\section{Conclusiones}
Teniendo en cuenta los simuladores anteriormente descritos, se puede extraer como conclusión que es deseable que un simulador de nivel de enlace o de sistema para 5G reúna como características la posibilidad de explotar al máximo las ventajas que 5G ofrece, como pueden ser la posibilidad de simular estaciones base de distinta frecuencia, movilidad de terminales, distintos tipos de celda, planificación geométrica, emparejamiento dinámico, simulaciones a muy altas frecuencias, alta disponibilidad -esto es, contar con un nodo de respaldo al que conectarse-, modelado de distintos tipos de entorno y de canal y modelado de antenas de las estaciones base mediante \acs{mimo}.

Si bien \textit{QuaDRiGa} no incluye todas estas características puesto que se concibe como un generador de canal en el que otros simuladores pueden basarse, como se explica en el Capítulo 4, se puede adelantar del mismo que las funcionalidades que debe incluir un simulador de un nivel superior a \textit{QuaDRiGa} son las de combinar escenarios con distintos tipos de celda, simulaciones multi-frecuencia, modelado de emparejamientos -incluidos los emparejamientos para terminales en movimiento- y obtención de parámetros concluyentes como cálculo de la capacidad del canal (\textit{troughput}) o del \ac{sinr}.
%
\chapter{Planificación y requisitos}\label{cap.requisitos}
\section{Introducción}
Bla bla bla
%
\chapter{Acerca de Quadriga}\label{cap.quadriga}
\section{Introducción}
El corazón de \textit{5Gneralife} está basado en \textit{QuaDRiGa} en su totalidad, por lo que resulta conveniente conocer previamente su funcionamiento y sus nociones básicas antes de detenerse a detallar la implementación del simulador en sí.

\textit{QUAsi Deterministic RadIo channel GenerAtor, QuaDRiGa} es un proyecto cuya finalidad es facilitar el desarrollo de simuladores de redes móviles de nivel de sistema, gracias a su función, que no es otra que generar respuestas impulsivas realistas para distintos canales de radio modelados de acuerdo a diversos estándares. En concreto, en su última versión hasta la fecha (v2.0), \textit{QuaDRiGa} integra un total de 78 modelos de canal, incluidos modelos de acuerdo a estándares creados por diversas entidades como 3GPP o WiNNER \cite{winner} o mmMAGIC.

Aunque \textit{QuaDRiGa} fue concebido para su uso en LTE, las actualizaciones más recientes han hecho compatible su uso con simulaciones orientadas a 5G mediante implementaciones como un rango más amplio de frecuencias, integración de MIMO en las antenas o inclusión de modelos de \textit{small cells} para los distintos entornos.

En este capítulo se especificarán sus principales características y especificaciones, así como su estructura de programación y su modo de uso.

\section{Características}

Puesto que \textit{QuaDRiGa} no pretende ser un simulador independiente, sino un recurso del que se pueden servir otros simuladores como el que ha sido desarrollado, no integra funcionalidades típicas de simuladores como datos característicos del entorno simulado.

Teniendo esto en cuenta, entre sus características generales, podemos destacar:

\begin{itemize}
    \item Permite configuración libre de capas de red con varios transmisores y receptores.
    \item Posibilidad de crear enlaces MIMO.
    \item Su rango de frecuencias es de 450 MHz hasta 100 GHz.
    \item Modela la evolución en el tiempo de sus elementos incluyendo parámetros de pequeña y gran escala.
    \item Modela escenarios rurales y urbanos, interiores y exteriores, todos ellos incluyendo los casos con visión directa (LOS) y sin visión directa (NLOS), con posibilidad de transición entre ellos por parte de los usuarios. Para estos modelos se hace uso de los parámetros establecidos por WINNER+ y por 3GPP-3D.
    \item Implementación completa para Matlab, con programación orientada a objetos. Compatibilidad con algunas versiones de Octave.
\end{itemize}

Una de las características más interesantes y que caracteriza a \textit{QuaDRiGa} es la posibilidad de implantación de distintos tipos de escenario en una misma simulación. Esta funcionalidad pretende modelar el comportamiento de un usuario que se encuentra en movimiento, ya sea a pie o en vehículo, ya que las condiciones de su entorno cambian constantemente a lo largo del recorrido.

\begin{figure}[h!]
	\centering
    \includegraphics[width=\linewidth]{imagenes/transiciones.png}
	\caption{Ilustración de un entorno de transiciones entre escenarios.}
	\label{fig:transicion}
\end{figure}

Para ejemplificar mejor el concepto de transiciones de la filosofía de implementación de \textit{QuaDRiGa}, se ha realizado una modificación de la figura de su documentación técnica \cite{quadrigadoc} visible en la Figura \ref{fig:transicion}. En ella se puede distinguir un total de seis segmentos cuya transición entre ellos se delimitan con los círculos blancos:

\begin{enumerate}
    \item En (1) se comienza el trayecto y se establece una conexión con la estación base con visión directa LOS en un entorno urbano.
    \item En el punto (2) se cambia de LOS a NLOS.
    \item En (3) se cambia de NLOS a LOS.
    \item Un cambio de dirección sin cambiar las condiciones de recepción (LOS).
    \item Se efectúa una parada con su consecuente cambio de velocidad mientras se mantienen las condiciones de visión directa.
    \item Se produce en (6) un cambio de entorno puesto que se accede a una zona rural. Las condiciones cambian de un escenario urbano con visión directa a uno rural sin visión directa (Urban LOS \(\rightarrow\) Rural NLOS).
    \item Antes de finalizar el trayecto, en (7) se produce un cambio de dirección sin cambiar las condiciones de no tener visión directa.
\end{enumerate}

Como se ha podido observar, este comportamiento típico de usuarios conlleva tener en cuenta varios cambios en las condiciones de recepción por parte del terminal, puesto que se deben considerar parámetros como el tipo de entorno, la visión directa, velocidad y  posición. \textit{QuaDRiGa} hace frente a esta problemática gracias a su modelado de tres dimensiones del entorno junto a la implementación de fusión de parámetros de canal generados, es decir, al generar los coeficientes de canal para un receptor, tiene en cuenta los cambios de tramos para que todos los coeficientes se encuentren correlados entre sí.

Sin embargo, el modelado de transiciones de escenario se encuentra intrínseco en el objeto dedicado al receptor, por lo que su utilidad se ve limitada a estudios de cambios de entorno -por ejemplo, cambio de entorno rural a urbano- pero al no ser un atributo propio de las estaciones base, esta funcionalidad no se puede utilizar para modelar distintos tipos de celda, por tanto, sería una tarea que el simulador de un nivel más alto debe implementar.

Lo que sí resulta útil para un simulador y que puede ser un atributo propio de las estaciones base y los terminales es la transición de LOS a NLOS y viceversa, con en el punto (2) y (3) de la Figura \ref{fig:transicion} puesto que \textit{QuaDRiGa} dispone de recursos que permiten simular la probabilidad de visión directa.

\section{Especificaciones técnicas}

Como se comentó anteriormente, \textit{QuaDRiGa} se sirve de Matlab como plataforma base de desarrollo. Esto ha hecho posible que la última versión hasta la fecha (v2.0.0) esté implementada totalmente con programación orientada a objetos, lo que permite una mayor flexibilidad, rendimiento y sencillez de uso.

Aunque la carga computacional de \textit{QuaDRiGa} puede resultar demasiado pesada según qué escenarios se quieran generar, sus requisitos mínimos de ejecución no resultan muy exigentes tal y como aparece en la Tabla \ref{tab:espec_quadriga}:

\begin{table}[h!]
\centering
\caption{Requisitos mínimos de QuaDRiGa}
\label{tab:espec_quadriga}
\begin{tabular}{c|c}
\textbf{Característica} & \textbf{Requisito mínimo} \\ \hline
Versión de Matlab       & 7.12 (R2011a)             \\
Toolbox                 & Ninguna                   \\
Memoria (RAM)           & 1 GB                      \\
Procesador              & 1 GHz un solo núcleo      \\
Almacenamiento          & 50 MB                     \\
Sistema operativo       & Linux, Windows, Mac OS   
\end{tabular}
\end{table}

Su instalación es sencilla, basta con añadir a Matlab el directorio en el que se encuentran los ficheros descargados con el comando \textit{addpath}:

\begin{lstlisting}[style=Matlab-editor, basicstyle=\tiny]
addpath('/[ruta hasta quadriga]/quadriga_src')
\end{lstlisting}

Una vez instalado, se puede hacer uso de sus funciones para crear entornos de simulación, como se explicará en el siguiente apartado.

\section{Visión de conjunto del software}

Esta sección está dedicada a ilustrar el funcionamiento de \textit{QuaDRiGa} plasmando así los resultados obtenidos de la fase de aprendizaje y toma de contacto con el generador. En primer lugar se detallarán sus pautas de uso, aclarando su estructura interna y el enfoque desde el que se debe usar. Seguidamente, se realizará una prueba para demostrar sus capacidades y plantear sus limitaciones, que serán la antesala y la motivación para el siguiente capítulo dedicado a la íntegra implementación del simulador.

\subsection{Estructura del software y utilización}

En primer lugar, es conveniente tener en consideración la estructura de software con la que cuenta \textit{QuaDRiGa}. Este generador implementa un total de siete clases para su procesamiento, cuatro de ellas implementan parámetros de entrada, dos están dedicadas a cómputos internos, y una última clase que engloban los resultados de salida.

De las clases que se utilizan como entrada podemos diferenciar:
\begin{itemize}
    \item \textbf{\textit{qd\_simulation\_parameters}} define ajustes generales como frecuencias y densidad de muestreo.
    \item \textit{\textbf{qd\_arrayant}} esta clase sirve para modelar las antenas que las estaciones base y los terminales móviles integran en sus comunicaciones, incluida la opción del modelaje MIMO.
    \item \textbf{\textit{qd\_track}} genera los objetos de seguimiento de terminales móviles. Sirve para modelar principalmente el movimiento de los usuarios y sus transiciones de escenario.
    \item \textbf{\textit{qd\_layout}} combina los seguimientos de receptores y los parámetros de simulación para generar capas de simulación independientes, una por cada frecuencia o tipo de estación base. Modela las posiciones de las estaciones base.
\end{itemize}

Por otro lado, en cuanto a las clases de procesamiento interno:

\begin{itemize}
    \item \textbf{\textit{qd\_sos}} se encarga de generar señales correladas entre sí basándose en el método conocido como \textit{suma de sinusoides}. Esta señales se utilizan para obtener parámetros sobre la señal recibida por parte de las estaciones móviles. 
    \item \textbf{\textit{qd\_builder}} crea los coeficientes de canal generando parámetros de gran escala y canales independientes para cada una de las antenas en el caso de usar modelado MIMO para las mismas.
\end{itemize}

Por último, la salida se modela con una séptima clase que engloba el resultado final derivado del uso del resto de las clases:

\begin{itemize}
    \item \textbf{\textit{qd\_channel}} contienen los datos de coeficientes de canal que se obtienen como resultado de las anteriores configuraciones. Esto incluye parámetros como amplitud y retardos de cada uno de las trayectorias del modelo de canal utilizado por \textit{QuaDRiGa}. También se encarga de fusionar las secuencias de muestras temporales del entorno para realizar una evolución temporal continua y coherente del escenario de simulación.
\end{itemize}

Cada una de las clases implementan atributos y métodos propios que por lo general sirven para generar a su vez objetos de otras clases o parámetros de salida. Además, todas las clases están perfectamente relacionadas entre sí, por lo que el mejor método para describir relaciones entre clases y los nombres de sus atributos y/o métodos es el uso de un diagrama como el de la Figura \ref{fig:uml_quadriga}, extraído de su propia documentación técnica \cite{quadrigadoc}.

% Diagrama UML

\begin{figure}[ht!]
	\centering
    \includegraphics[width=\linewidth]{imagenes/uml_quadriga.png}
	\caption{Diagrama de clases UML de QuaDRiGa.}
	\label{fig:uml_quadriga}
\end{figure}

% Pasos de uso

Cada simulación en \textit{QuaDRiGa} está hecha en tres pasos más un cuarto paso opcional, obteniendo como resultado un objeto de la clase \textit{qd\_channel}:

\begin{enumerate}
    \item Configurar escenario. Esto conlleva la declaración de objetos de las cuatro clases que se toman como entrada, los cuales modelan antenas -tanto para receptor como emisor, para lo cual \textit{QuaDRiGa} ya incluye sus propios modelos predeterminados-, seguimiento de terminales móviles -especificando patrón de movimiento, distancia recorrida, etc.-, escenarios para terminales y/o estaciones base y, por último, capas de simulación que, como se mencionó anteriormente, están caracterizados por parámetros como frecuencia de trabajo, posiciones de estaciones base o emparejamientos entre estaciones base y terminales.
    \item Generar parámetros de gran escala correlados. Esto se hace mediante el uso de la clase \textit{qd\_sos} que a través de los parámetros extraídos de la base de datos de canales que integra \textit{QuaDRiGa}, genera señales aleatorias a través del método de suma de sinusoides.
    \item Calcular los coeficientes de canal fusionados y correlados a través de la clase \textit{qd\_builder} que unifica todos los objetos de entrada más las señales generadas en el paso número 2 para obtener coeficientes de canal representados mediante sus amplitudes, desfases y otros parámetros como ángulos de llegada. Además, se ofrece la opción de obtener dichos coeficientes en dominio de la frecuencia para posibles procesados posteriores.
    \item Post-procesado (opcional). Dentro de los procesados posteriores, más allá del método que implementa la clase dedicada a los canales, no existen más herramientas que permitan obtener resultados más representativos o concluyentes sobre la simulación. Por ello, en este cuarto paso es donde se centra la labor de desarrollo del simulador, como se verá en el Capítulo \ref{cap.implementacion}.
\end{enumerate}

\subsection{Capacidades y limitaciones. Demostración práctica}

Una vez que se adquiere una visión general sobre \textit{QuaDRiGa}, su uso y los conceptos que engloba, es posible realizar un uso estándar del mismo. En esta sub-sección se realizará un ejemplo de simulación utilizando únicamente \textit{QuaDRiGa} con la finalidad de plasmar sus funcionalidades y sus carencias, de modo que podrá ser comparado con futuros ejemplos de uso de \textit{5Gneralife}.

En primer lugar, señalar su uso exclusivo mediante línea de comandos y/o \textit{scripts}, lo cual puede resultar abstracto para un usuario que comience a usar \textit{QuaDRiGa}, especialmente en casos en los que dicho usuario nunca antes haya realizado una toma de contacto con un simulador de comunicaciones móviles.

Como solamente se puede controlar por línea de texto, para este ejemplo se ha desarrollado un \textit{script} que pretende simular una capa de simulación para un entorno de macro-celdas completamente urbano. Dicha capa contará con 10 receptores en movimiento, recorriendo cada uno 1 km. Además, se incluirá un total de tres estaciones base. Tanto terminales como estaciones base tendrán una posición aleatoria y sus antenas serán omnidireccionales de ganancia 5 dBi para simplificaciones. La frecuencia de trabajo será de 2,6 GHz.

Para la puesta en marcha del escenario, el primer paso es la configuración de parámetros de simulación, tarea que solamente precisa de crear un objeto de la clase \textit{qd\_simulation\_parameters} y modificar su atributo de frecuencia a 2,6 GHz:

\begin{lstlisting}[style=Matlab-editor, basicstyle=\tiny]
%% Parametros de simulacion
sim_param = qd_simulation_parameters; % Se genera un objeto de parametros
sim_param.center_frequency = 2.6e9;   % Frecuencia de 2,6 GHz
\end{lstlisting}

A continuación, se hace el proceso análogo para crear dos objetos de la clase \textit{qd\_arrayant} que modelarán sendas antenas omnidireccionales, una para cada tipo de elemento de la simulación -terminales y estaciones base-:

\begin{lstlisting}[style=Matlab-editor, basicstyle=\tiny]
%% Antenas
antenaMT = qd_arrayant('omni'); % Asignacion de antena omnidireccional de ganancia 5 dBi
antenaBS = qd_arrayant ('omni');
\end{lstlisting}

Acto seguido, se realiza la elaboración de la capa de simulación. Como en este caso no existe más de un tipo de celda ni más de una frecuencia, solo es necesario crear una capa. La capa integra los datos sobre estaciones base y receptores, así como el tipo de entorno. En primer lugar se crea el objeto y se configuran los emisores y receptores con el número de cada uno de ellos y sus corresponientes antenas:

\begin{lstlisting}[style=Matlab-editor, basicstyle=\tiny]
%% Transmisores y receptores

layout_uma = qd_layout(sim_param); % Generamos capa de simulacion
layout_uma.no_tx = 3; % 3 estaciones base
layout_uma.no_rx = 10; % 10 usuarios

layout_uma.rx_array = antenaMT; % Asignacion de antena de terminal
layout_uma.tx_array = antenaBS; % Asignacion de antena de BS
\end{lstlisting}

Acto seguido, se configuran las posiciones de los mismos. Para ello se hace uso del método \textit{randomize\_rx\_positions} para los receptores, y un pequeño bucle \textit{for} para los transmisores. Se ha decidido que la distancia máxima del centro a la que pueden estar los elementos sea de 2 km:

\begin{lstlisting}[style=Matlab-editor, basicstyle=\tiny]
%% Posiciones
% Posiciones de usuarios aleatorias con altura de 1,5 m:
layout_uma.randomize_rx_positions(2000, 1.5, 1.5, 0); 

% Posiciones de Bs aleatorias:
for i = 1:layout_uma.no_tx
    pos_x = randi(4000) - 2000; % Posicion X aleatoria entre -2000 y 2000 m del centro
    pos_y = randi(4000) - 2000; % Posicion Y aleatoria entre -2000 y 2000 m del centro
    altura = 25; % 25 metros de altura
    layout_uma.tx_position(:, i) = [pos_x, pos_y, altura];
end
\end{lstlisting}

Para completar la configuración de la capa, es necesario modelar los dos últimos aspectos del escenario: el entorno y el movimiento de los usuarios. Para el movimiento, se ha utilizado un modelado de calle que \textit{QuaDRiGa} implementa, con una orientación aleatoria entre 0 y 360 grados, una longitud mínima de 50 m para las calles, una media de longitud de calle de 187 m, una desviación típica de 83 metros para su longitud, un radio de curva de 10 metros para los giros y una probabilidad de girar en un cruce de 0,5 -según la documentación de \textit{QuaDRiGa} es la son los datos obtenidos para modelar las calles de Berlín-. Para el tipo de entorno, se utiliza el método \textit{set\_scenario} tanto para el objeto de \textit{layout} como para cada uno de los objetos \textit{track}, especificando un tipo de entorno de acuerdo al estándar de macro-celda urbana especificado por 3GPP-3D \cite{3gpp3d}, con una probabilidad de visión directa de 0,2, extraído del mismo:

\begin{lstlisting}[style=Matlab-editor, basicstyle=\tiny]
%% Modelado del escenario
NLOS_prob = 0.8; % Probabilidad de NLOS de 80%
layout_uma.set_scenario('3GPP_3D_UMa',[],[]); % Asignacion de macro-celda segun 3GPP-3D

% Generamos movimientos para los usuarios con modelado de calle de Berlin
trk = cell(1);
for a = 1 : layout_uma.no_rx
    trk{1} = qd_track('street', 1000, randi(360), 50, 187, 83, 10, 0.5); % Calle
    trk{1}.initial_position = layout_uma.rx_position(:,a); % Posicion de acuerdo a la inicial
    trk{1}.name = layout_uma.rx_name{a}; % Se asigna un nombre unico
    escenarios = {'3GPP_3D_UMa_LOS', '3GPP_3D_UMa_NLOS'}; % Se asignan escenarios
    trk{1}.set_scenario( escenarios, [1-NLOS_prob, NLOS_prob], [] ); % Se crean segmentos
    layout_uma.track(1,a) = trk{1}.copy; % Se asigna a la capa
end

\end{lstlisting}

Para obtener las variables de salida, el único paso que queda es el de crear un objeto de la clase \textit{qd\_builder} con la finalidad de generar efectos de pequeña escala y finalmente, generar los coeficientes de canal y unirlos para que entre ellos exista correlación. Cabe destacar que el hecho de generar los efectos de pequeña escala aumenta considerablemente el tiempo de ejecución pero permite tener una simulación sustancialmente más realista.

\begin{lstlisting}[style=Matlab-editor, basicstyle=\tiny]
%% Generamos canales
builder = layout_uma.init_builder; % Objeto builder para pequena escala
gen_ssf_parameters( builder ); % Generamos efectos de pequena escala
canales = get_channels( builder ); % Generamos coeficientes de canal
canales_finales = merge( canales );
\end{lstlisting}

Por último, se puede visualizar la capa a través del método que \textit{QuaDRiGa} ofrece, obteniendo como resultado una representación como la de la Figura \ref{fig:repres_ejemplo}.

\begin{lstlisting}[style=Matlab-editor, basicstyle=\tiny]
%% Visualizacion
layout_uma.visualize([],[],2);
\end{lstlisting}

\begin{figure}[h!]
	\centering
    \includegraphics[width=\linewidth]{imagenes/visualizacion_ejemplo.png}
	\caption{Visualización del entorno simulado proporcionada por \textit{QuaDRiGa}.}
	\label{fig:repres_ejemplo}
\end{figure}

Como se observa en la Figura \ref{fig:repres_ejemplo}, se trata de una representación en tres dimensiones que resulta bastante simple a la vez que no detalla ciertos aspectos. Las líneas rojas representan las estaciones base mientras que el cúmulo de puntos es el modo que tiene \textit{QuaDRiGa} de mostrar las trayectorias de sus receptores en movimiento. Si ampliamos la zona de uno de los receptores, se puede diferenciar la línea del recorrido mientras que cada uno de los círculos azules representa un cambio de segmento. En cada transición de segmento se asigna un escenario aleatorio con probabilidad 0.2 para LOS y 0.8 para NLOS. Véase la Figura \ref{fig:receptor_ejemplo}.

\begin{figure}[h!]
	\centering
    \includegraphics[width=\linewidth]{imagenes/visualizacion_ejemplo_rx1.png}
	\caption{Visualización del recorrido de uno de los receptores.}
	\label{fig:receptor_ejemplo}
\end{figure}

Además, resulta interesante efectuar una indagación en el objeto de la clase \textit{qd\_channel} que se ha generado como salida. Para empezar, la variable \textit{canales\_finales} es un \textit{array} de celdas que cuenta con unas dimensiones de 1x30. Existe un objeto de tipo canal para cada uno de los posibles enlaces, esto es, para cada uno de los 10 receptores, existen 3 objetos de canal, uno `por cada estación base.

Si se abren los datos de uno de los 30 canales, se puede observar la composición del mismo, como muestra la Figura \ref{fig:variable_canal}:

\begin{figure}[h!]
	\centering
    \includegraphics{imagenes/canal_generado_ejemplo.PNG}
	\caption{Estructura de un objeto de la clase \textit{qd\_channel} generado en el ejemplo.}
	\label{fig:variable_canal}
\end{figure}

En primer lugar, se aprecia el nombre del canal que hace referencia a los dos elementos que constituyen el enlace, el transmisor número 1 -Tx01- y el receptor número 1 -Rx01-. Entre los atributos de interés se encuentran las variables \textit{coeff} y \textit{delay}, dos matrices de dimensiones 1x1x34x1000 -dimensiones correspondientes al número de receptores, el número de emisores, el número de trayectorias que el rayo de la señal ha tomado, y el número de posiciones del receptor a través de su recorrido- que contienen las amplitudes y las fases de los coeficientes de canal generados, normalizados. Estos datos a priori no proporcionan información tangible para simulaciones, por lo que es tarea de un simulador basado en \textit{QuaDRiGa} la de procesar posteriormente estos canales con la finalidad de extraer información y obtener datos de caracterización específicos, como podría ser la capacidad del canal.

Como dato a tener en cuenta, el proceso de la simulación de este ejemplo tomó un tiempo de ejecución de 209,25 segundos -3 minutos y medio aproximadamente- utilizando un ordenador de las características que aparecían en la Tabla \ref{tab:caracteristicas}, por lo que es de esperar que para simulaciones más complejas, sean necesarias unas capacidades computacionales considerables.

Para concluir y a modo de resumen, mencionar que para ilustrar el funcionamiento de \textit{QuaDRiGa} con un ejemplo, se ha configurado una entrada compuesta por la generación de objetos que han parametrizado los detalles de la simulación -como por ejemplo la frecuencia de trabajo-. También la capa de simulación incluyendo en ella el número de terminales y su movimiento, el número de estaciones base y su posición, así como el entorno de simulación -en concreto, se ha optado por el modelo 3D de 3GPP publicado en el informe técnico 36.873-. Por otro lado, como salida se ha obtenido un objeto de la clase \textit{qd\_channel}, cuya arquitectura permite obtener las amplitudes normalizadas y las fases de la señal de enlace entre cada uno de los terminales y las estaciones base. También se ha podido obtener una representación visual de la capa de simulación (Figura \ref{fig:esquema_ejemplo}).

% Figura de un esquema
\begin{figure}[ht!]
	\centering
    \includegraphics[width=\linewidth]{imagenes/diagrama_ejemplo.png}
	\caption{Diagrama de utilización de \textit{QuaDRiGa}.}
	\label{fig:esquema_ejemplo}
\end{figure}

Si bien estos parámetros de salida son los más dificultosos de obtener debido a su complejidad de cómputo y a la estricta normativa de estándares por los que se rige, no son suficientes y necesitan un procesado extenso para la obtención de datos propios de un simulador. Por ello, el reto que se plantea para implementar un simulador a partir de \textit{QuaDRiGa} como el que se ha desarrollado para este Trabajo de Fin de Grado es el de un código que, de por sí solo, sea capaz de generar e interpretar coeficientes de canal que surjan a partir de configuraciones específicas de 5G, incluyendo, entre otras, funcionalidades de creación de celdas de distinta naturaleza, compartir los mismos usuarios entre todas las capas de simulación -algo que no es posible con \textit{QuaDRiGa}-, obtención de visualizaciones mejoradas y cálculo de parámetros característicos de salida que permitan evaluar objetiva y fácilmente la red creada.


%
\chapter{Diseño e implementación}\label{cap.implementacion}
\section{Introducción}
Bla bla bla
%
\chapter{Análisis y simulaciones de prueba}\label{cap.pruebas}

La naturaleza de este proyecto no solo contempla la implementación de un código que sea capaz de realizar simulaciones de entornos 5G, sino que también está orientado a ofrecer resultados de simulaciones realizadas, y orientaciones sobre cómo efectuar simulaciones apropiadas para el paradigma de 5G y \acs{hetnet}s.

Por ello, este capítulo está dedicado a la exposición de una serie de simulaciones realizadas a través de 5Gneralife, con la finalidad de evaluar el comportamiento de las redes 5G en distintas condiciones, a través de la variación progresiva de parámetros, así como valorar el rendimiento del mismo, tiempos de ejecución y recursos computacionales consumidos.

Para comenzar, se realizará en una primera sección una descripción de los detalles que se deben tener en cuenta y los parámetros recomendados a la hora de realizar una simulación realista. A continuación, se realizará una demostración de un entorno realista siguiendo dichas pautas. Por último, se llevarán a cabo un conjunto de simulaciones experimentales para evaluar cambios en el comportamiento de escenarios 5G.

\section{Estándar propuesto}

Aunque todas las implementaciones que se han llevado a cabo para la elaboración de este simulador han sido orientadas hacia la funcionalidad de 5G, no hay que olvidar que el software en el que se basa, \textit{QuaDRiGa}, fue diseñado con vistas en LTE, con posterior adaptación a 5G. Esto supone que la elección de parámetros deben ser consecuentes con el escenario deseado y, por tanto, hay que prestar especial atención a ciertos detalles de la naturaleza de 5G.

Por ello, está sección está dedicada a recopilar los detalles que se deben tener en cuenta a la hora de realizar simulaciones que se esperan que sean realistas, junto a parámetros o rango de parámetros recomendados para que el escenario sea lo más ajustado posible a 5G.

A continuación, se detalla un listado de consideraciones:
\begin{itemize}
    \item Las macro-celdas se encuentran para auxiliar a las micro-celdas y no al revés, como se concebía en LTE. Esto implica que se debe desplegar un entorno en el que se espera que la mayor concentración de usuarios se den en los enlaces hacia micro-celdas, por lo que es necesario dimensionarlas de forma que el enlace por SINR -o, en su defecto, por distancia- sea favorable siempre en la capa de micro-celdas con respecto a la capa de macro-celdas. Por ello, se propone que la distancia entre micro-celdas sea aproximadamente la mitad que la distancia entre macro-celdas establecida, a la vez que la potencia de transmisión recomendada sea de 40 dBm para macro-celdas y 27 dBm para micro-celdas. Con esta configuración se han obtenido unos resultados óptimos de cobertura.
    \item Un escenario con una relación cobertura-proporción de número de elementos desplegados razonable es la de despliegue de 7 macro-celdas y 19 micro-celdas para ofrecer una cobertura optimizada de acuerdo a la separación entre sus centros establecida anteriormente.
    \item El modelo de ruido térmico típico para 5G se suele especificar con una densidad espectral de $4.04·10^{-21}$ W/Hz \cite{ruido}.
    \item La probabilidad de encontrar visión directa según el reporte técnico de 3GPP 38.901 se encuentra en 20\%, correspondiendo un 80\% para el caso de NLOS de acuerdo a los estudios presentados en su última versión.
    \item La frecuencia estándar para macro-celdas urbanas es de aproximadamente 2,6 GHz, mientras que para micro-celdas sería de 20 GHz.
    \item No es realista que todos los usuarios realicen una trayectoria sin final. Por este motivo y también para evitar simulaciones de larga duración, lo razonable es que una persona recorra una media de 2 km mientras se desplaza por la calle.
\end{itemize}

Establecidas estas recomendaciones, comentar que \textit{5Gneralife} permite cualquier configuración en la medida de lo razonable y que dependiendo del número de usuarios, esta simulación puede llevarse a cabo o no, debido a que una gran cantidad de usuarios puede llegar a consumir todos los recursos de memoria y aumenta considerablemente el tiempo de ejecución.

\section{Estudio completo de una sola simulación}

En primer lugar para ilustrar el potencial del simulador, se ha efectuado una simulación consistente en un entorno con 100 usuarios en movimiento. Cada usuario recorre 250 metros a una velocidad de un metro por segundo y se encuentra utilizando un terminal móvil con interfaz dual compatible con macro-celdas y micro-celdas.

La frecuencia de las celdas son de 2,6 GHz para el caso de las macro-celdas y de 20 GHz para micro-celdas. Además, los centros de las macro-celdas se encuentran separados entre sí por 3 km, mientras que los de las micro-celdas están separados por 1,3 km. La potencia de transmisión son de 40 dBm y 27 dBm para macro-celdas y micro-celdas respectivamente.

Se ha adoptado un ancho de banda de 100 MHz por celda y un ruido térmico con una densidad espectral de $4.04·10^{-21}$ W/Hz. 

Se puede ver una representación gráfica del resultado final en la figura \ref{fig:simulacion_completa_entorno}:

\begin{figure}[h!]
	\centering
    \includegraphics[width=0.8\linewidth]{imagenes/6_2_entorno.png}
	\caption{Visualización del entorno simulado.}
	\label{fig:simulacion_completa_entorno}
\end{figure}

Esta simulación ha tomado un tiempo de ejecución de 5 horas, 28 minutos y 3 segundos con el equipo descrito en la sección de recursos. Para la simulación se ha llegado a requerir un máximo de 3 GB de memoria RAM, y la media de utilización de CPU ha sido del 33\% a lo largo de toda su ejecución. Para el almacenamiento de los datos generados se han necesitado 17,80 MB de espacio. Se estima que si esta simulación hubiese contado con un recorrido por parte de los usuarios de 1 km, esta habría tardado aproximadamente 27 horas.

Para mostrar unos resultados significativos, se ha considerado comparar las variables de SINR y capacidad de los cuatro primeros receptores para las dos capas de simulación en un total de cuatro figuras.

En las dos primeras, Figura \ref{fig:simulacion_completa_cap_uma} y Figura \ref{fig:simulacion_completa_cap_umi} se puede comparar la capacidad de salida para dichos usuarios en el caso de la capa de macro-celdas y de micro-celdas respectivamente:

\begin{figure}[h!]
	\centering
    \includegraphics[width=0.8\linewidth]{imagenes/6_2_capacidad_uma.png}
	\caption{Capacidad de canal para los primeros cuatro usuarios en la capa de macro-celda.}
	\label{fig:simulacion_completa_cap_uma}
\end{figure}

\begin{figure}[h!]
	\centering
    \includegraphics[width=0.8\linewidth]{imagenes/6_2_capacidad_umi.png}
	\caption{Capacidad de canal para los primeros cuatro usuarios en la capa de micro-celda.}
	\label{fig:simulacion_completa_cap_umi}
\end{figure}

Se observa una capacidad asignada significativamente mayor para el caso de micro-celdas. Este resultado es de esperar si se tiene en cuenta que el número de usuarios conectados en la capa de micro-celdas a una misma estación base siempre es menor que en el caso de la capa de macro-celdas, por lo que el ancho de banda es repartido entre menor cantidad de usuarios.

\begin{figure}[h!]
	\centering
    \includegraphics[width=0.8\linewidth]{imagenes/6_2_sinr_uma.png}
	\caption{SINR para los primeros cuatro usuarios en la capa de macro-celda.}
	\label{fig:simulacion_completa_sinr_uma}
\end{figure}

\begin{figure}[h!]
	\centering
    \includegraphics[width=0.8\linewidth]{imagenes/6_2_sinr_umi.png}
	\caption{SINR para los primeros cuatro usuarios en la capa de micro-celda.}
	\label{fig:simulacion_completa_sinr_umi}
\end{figure}

Por otro lado, en cuanto a SINR, como se aprecia en las Figuras \ref{fig:simulacion_completa_sinr_uma} y \ref{fig:simulacion_completa_sinr_umi}, aunque los resultado se decantan por ser ligeramente propicios en el caso de las micro-celdas, la diferencia no es sumamente relevante. Esto demuestra que el factor que hacía que la capacidad de canal asignada fuera mayor en micro-celdas era el reparto de usuarios en las celdas. Esto justifica la existencia de las micro-celdas en generaciones avanzadas de radio-comunicaciones, ya que su funcionalidad es precisamente descargar de usuarios a las celdas más grandes.

Del mismo modo, es posible comparar los dos criterios de emparejamiento disponibles a través de la comparativa entre la capacidad en ambos casos. Para ello, se ha representado en una misma figura la evolución de la capacidad en los dos casos de emparejamiento, como aparece en la Figura \ref{fig:simulacion_completa_sinr_umi_vs_cercano}.

\begin{figure}[h!]
	\centering
    \includegraphics[width=0.8\linewidth]{imagenes/6_2_capacidad_umi_sinr_vs_cercano.png}
	\caption{Comparativa de la capacidad del primer receptor para los dos criterios de emparejamiento.}
	\label{fig:simulacion_completa_sinr_umi_vs_cercano}
\end{figure}

En ella se muestra que, por lo general, el criterio de emparejamiento de mayor SINR es por lo general mejor. Sin embargo, hay tramos en los que ocurren excepciones y que demuestran que hay ocasiones en las que es mejor para el rendimiento de la red asignar los usuarios a otras celdas que no sean las óptimas, ya que en casos de congestión existirá saturación.

Por último, podría resultar también interesante consultar cómo evoluciona la capacidad asignada de acuerdo con la SINR recibida. En la Figura \ref{fig:simulacion_completa_sinr_uma_vs_capacidad} se muestra a la vez la SINR y la capacidad del primer usuario a lo largo de su recorrido.

\begin{figure}[h!]
	\centering
    \includegraphics[width=0.8\linewidth]{imagenes/6_2_capacidad_vs_sinr_uma.png}
	\caption{Evolución de la SINR y de la capacidad para el primer usuario.}
	\label{fig:simulacion_completa_sinr_uma_vs_capacidad}
\end{figure}

De los datos de simulación generados se puede extraer con una evaluación breve que el sistema se comporta tal y como se esperaba, con resultados proporcionados, un comportamiento realista en cuanto a rendimiento de la red se refiere y con una completa cohesión entre las salidas.

\section{Comparativa entre simulaciones}

No solamente es interesante el estudio de un escenario complejo de simulación como el anterior, sino que también resulta conveniente realizar las oportunas pruebas para estudiar cómo cambia el comportamiento de las variables de salida al alterar diversos parámetros de simulación.

Por ello, como último conjunto de pruebas, se han realizado tres estudios independientes con la finalidad de evaluar la repercusión en ambas camas de simulación para tres parámetros distintos: frecuencia, potencia de transmisión y separación entre centros de celdas.

Para las simulaciones se ha establecido un recorrido fijo para un único receptor, el cual recorre 1.000 metros a lo largo de un entorno con 7 macro-celdas y 19 micro-celdas. Se ha fijado el ancho de banda de cada celda a 100 MHz y el ruido con una densidad de $4.04·10^{-21}$ W/Hz. En concreto, el recorrido que realiza el receptor es el mostrado en la Figura \ref{fig:recorrido_pruebas}.

\begin{figure}[h!]
	\centering
    \includegraphics[width=0.8\linewidth]{imagenes/representacion_pruebas_entorno.png}
	\caption{Entorno de simulación de pruebas con el recorrido del único receptor.}
	\label{fig:recorrido_pruebas}
\end{figure}

En dicha representación se puede observar que el receptor se encuentra en una posición a poca distancia del centro de la primera celda en el caso de macro-celdas, mientras que el escenario para las micro-celdas es algo menos propicio, puesto que se aprecia que el receptor pasa de una celda a otra, encontrándose en todo momento en la frontera entre la celda 2 y la celda 3 de dicha capa, por lo que es de esperar que las comunicaciones en la capa de celdas no sean las óptimas en casi todas las simulaciones.

Aunque los resultados pueden otorgar información de cómo afecta al entorno, este número de simulaciones no es suficiente para resultar concluyente, puesto que las condiciones tienen una componente aleatoria que puede propiciar las interferencias en un determinado caso entre otros aspectos conflictivos. La mejor metodología a seguir sería repetir la simulación de la anterior sección realizando un barrido paramétrico para cada una de las variables, y recopilando información sobre los valores medios de salida para todos los receptores. Sin embargo, esta labor podría llevar miles de horas con un solo ordenador como el que se ha utilizado, por lo que se ha optado por esta simplificación que puede resultar de igual modo significativa.

\subsection{Evolución de la frecuencia}

El primer parámetro que se variará para evaluar el comportamiento del escenario es la frecuencia. Para realizar el experimento, se han diseñado tres casos en los que únicamente varía la frecuencia asignada a los dos tipos de celda en cada uno de ellos, manteniendo estático el resto de parámetros.

En cuanto a los parámetros estáticos, se ha diseñado una simulación con un receptor que recorre 1.000 metros. La capa de macro-celdas está compuesta por 7 celdas, mientras que la de las micro-celdas, está compuesta por 19, con una separación entre sus centros de 5 km y 2,2 km respectivamente. Se ha mantenido el ruido fijo con una densidad espectral de potencia de $4,04·10^{-21}$ W/Hz, y un ancho de banda de 100 MHz para ambas capas. La potencia de transmisión se ha fijado a 40 dBm para macro-celdas y 27 dBm para micro-celdas.

En primer lugar, se ha realizado una simulación inicial con una frecuencia que se estima como mínima, puesto que se ha asignado 1 GHz para macro-celdas y 6 GHz para las micro-celdas. Para este caso, se ha representado la capacidad obtenida por el receptor en el caso de conexión a macro-celdas en la Figura \ref{fig:simulacion_freq_min_uma} y en el caso de conexión a micro-celdas en la Figura \ref{fig:simulacion_freq_min_umi}, ambas para el caso de emparejamiento de la estación base con mayor SINR.

Esta simulación ha tomado un total de 21 minutos y 44 segundos para su ejecución, obteniendo un pico de utilización de memoria de 1 GB y ocupando 7,88 MB de almacenamiento al ser guardada en el disco duro.

\begin{figure}[h!]
	\centering
    \includegraphics[width=0.8\linewidth]{imagenes/6_3_capacidad_uma_minimo.png}
	\caption{Capacidad del receptor en la capa de macro-celda con una frecuencia baja.}
	\label{fig:simulacion_freq_min_uma}
\end{figure}

\begin{figure}[h!]
	\centering
    \includegraphics[width=0.8\linewidth]{imagenes/6_3_capacidad_umi_minimo.png}
	\caption{Capacidad del receptor en la capa de micro-celda con una frecuencia baja.}
	\label{fig:simulacion_freq_min_umi}
\end{figure}

En las gráficas resultantes por parte de esta simulación se aprecian unos valores aceptables para la capacidad asignada en la capa de macro-celdas, con picos de aproximadamente 450 Bit/s a pesar de haber algún tramo sin cobertura, mientras que para la simulación de la capa de micro-celdas, aunque el valor máximo se asemeja, existen numerosos tramos en los que no existe cobertura y la capacidad, por tanto, es nula. Esto puede deberse a la proximidad de las celdas y su baja frecuencia. Al bajar la frecuencia, el desvanecimiento de la señal por propagación es menor, lo que conlleva que haya una mayor cantidad de interferencias que limitan el rendimiento de la red.

Acto seguido, se realiza la misma simulación, esta vez modificando las frecuencias de la estaciones base a valores de 2,6 GHz para el caso de las macro-celdas y a 20 GHz para el caso de las micro-celdas. Estas suposiciones resultan lo más cercano al estándar. Los resultados son visibles en las Figuras \ref{fig:simulacion_freq_med_uma} y \ref{fig:simulacion_freq_med_umi}.

Esta simulación ha tomado un total de 20 minutos y 26 segundos para su ejecución, obteniendo un pico de utilización de memoria de 1 GB y ocupando 7,88 MB de almacenamiento al ser guardada en el disco duro.

\begin{figure}[h!]
	\centering
    \includegraphics[width=0.8\linewidth]{imagenes/6_3_capacidad_uma_medio.png}
	\caption{Capacidad del receptor en la capa de macro-celda con una frecuencia media.}
	\label{fig:simulacion_freq_med_uma}
\end{figure}

\begin{figure}[h!]
	\centering
    \includegraphics[width=0.8\linewidth]{imagenes/6_3_capacidad_umi_medio.png}
	\caption{Capacidad del receptor en la capa de micro-celda con una frecuencia media.}
	\label{fig:simulacion_freq_med_umi}
\end{figure}

En este caso, la situación para la capa de macro-celdas no varía sustancialmente, viéndose reducida la capacidad asignada a costa de una conexión más estable. Por otro lado, para las micro-celdas, existen tramos con mejor cobertura que en la anterior simulación, aunque sigue siendo muy inestable con tramos de cobertura nula que se podrían justificar por encontrarse en la frontera entre dos celdas.

Para finalizar, se simula un último escenario, esta vez incrementando las frecuencias sustancialmente. Para el caso de la capa de macro-celdas, se ha elegido una frecuencia central de 10 GHz mientras que para la capa de micro-celdas, se ha establecido la frecuencia a 60 GHz. La representación de la capacidad para el receptor en cada caso puede consultarse en las Figuras \ref{fig:simulacion_freq_max_uma} y \ref{fig:simulacion_freq_max_umi}, respectivamente.

Esta simulación ha tomado un total de 20 minutos y 18 segundos para su ejecución, obteniendo un pico de utilización de memoria de 1 GB y ocupando 7,88 MB de almacenamiento al ser guardada en el disco duro.

\begin{figure}[h!]
	\centering
    \includegraphics[width=0.8\linewidth]{imagenes/6_3_capacidad_uma_maximo.png}
	\caption{Capacidad del receptor en la capa de macro-celda con una frecuencia alta.}
	\label{fig:simulacion_freq_max_uma}
\end{figure}

\begin{figure}[h!]
	\centering
    \includegraphics[width=0.8\linewidth]{imagenes/6_3_capacidad_umi_maximo.png}
	\caption{Capacidad del receptor en la capa de micro-celda con una frecuencia alta.}
	\label{fig:simulacion_freq_max_umi}
\end{figure}

Para esta última simulación, la cobertura se ve ligeramente reducida para el caso de la macro-celda, mientras que resulta que para la capa de micro-celda, la cobertura se ve beneficiada y la capacidad aumenta con respecto a simulaciones de menor frecuencia, probablemente por el hecho de recibir menor cantidad de interferencias gracias al desvanecimiento.

Por tanto, se puede concluir que para este escenario en concreto, la configuración óptima sería asignar una frecuencia de 2,6 GHz a la capa de macro-celdas y de 60 GHz a la capa de micro-celdas.

Sin embargo, la principal ventaja de elevar la frecuencia es la de poder ampliar el ancho de banda, característica que no está modelada en este estudio debido a que la implementación del ancho de banda no se ha tenido en cuenta para ser independiente en el caso de macro-celdas y micro-celdas.

Por ello, es de esperar que a medida que se eleva la frecuencia, el ancho de banda asignado también crezca, de modo que, consecuentemente, la capacidad de canal se vea significativamente aumentada.

\subsection{Evolución de la potencia de transmisión}

El siguiente parámetro a evaluar es la potencia de transmisión. Este parámetro influye en la señal recibida, tanto para la estación base conectada como para el resto de estaciones base que interfieren. Por ello, el estudio de cómo afecta la potencia al escenario puede resultar relevante para extraer datos sobre qué potencia de transmisión es la óptima para evitar interferencias a la vez que se recibe una buena calidad de señal.

Para el experimento, se han establecido tres simulaciones que tienen como parámetros comunes la cantidad de celdas, 7 macro-celdas y 19 micro-celdas en concreto, con frecuencias de trabajo de 2,6 GHz y 20 GHz respectivamente y una separación entre centros de 5 km y 2,2 km. El usuario realiza el mismo recorrido que en las anteriores simulaciones y se ha mantenido el ruido fijo con una densidad espectral de potencia de $4,04·10^{-21}$ W/Hz, y un ancho de banda de 100 MHz para ambas capas.

La primera simulación se ha realizado con una potencia de transmisión de 27 dBm para las macro-celdas y de 20 dBm para el caso de las micro-celdas. Se ha representado gráficamente la capacidad del usuario a lo largo de su recorrido, en ambas capas de simulación, que pueden verse en las Figuras \ref{fig:simulacion_pot_min_uma} y \ref{fig:simulacion_pot_min_umi}.

Esta simulación ha tomado un total de 20 minutos y 25 segundos para su ejecución, obteniendo un pico de utilización de memoria de 1 GB y ocupando 7,88 MB de almacenamiento al ser guardada en el disco duro.

\begin{figure}[h!]
	\centering
    \includegraphics[width=0.8\linewidth]{imagenes/6_4_capacidad_uma_minimo.png}
	\caption{Capacidad del receptor en la capa de macro-celda con una potencia de transmisión baja.}
	\label{fig:simulacion_pot_min_uma}
\end{figure}

\begin{figure}[h!]
	\centering
    \includegraphics[width=0.8\linewidth]{imagenes/6_4_capacidad_umi_minimo.png}
	\caption{Capacidad del receptor en la capa de micro-celda con una potencia de transmisión baja.}
	\label{fig:simulacion_pot_min_umi}
\end{figure}

Se puede considerar que la estabilidad de la red es aceptable aunque no es la óptima, puesto que la capacidad de red no llega a valores máximos que en otros anteriores escenarios se alcanzaban. Existen algunas zonas sin cobertura, especialmente para la capa de micro-celdas.

Se repite el proceso, esta vez modificando los valores de potencia de transmisión a 40 dBm para macro-celdas y 27 dBm para micro-celdas, obteniendo unos resultados similares a los de la anterior sub-sección, puesto que son los datos estándar, dichos resultados aparecen en la Figura \ref{fig:simulacion_pot_med_uma} y en la Figura \ref{fig:simulacion_pot_med_umi}.

Esta simulación ha tomado un tiempo de ejecución de 21 minutos y 12 segundos, con una ocupación máxima de 1 GB de memoria RAM.

\begin{figure}[h!]
	\centering
    \includegraphics[width=0.8\linewidth]{imagenes/6_4_capacidad_uma_medio.png}
	\caption{Capacidad del receptor en la capa de macro-celda con una potencia de transmisión media.}
	\label{fig:simulacion_pot_med_uma}
\end{figure}

\begin{figure}[h!]
	\centering
    \includegraphics[width=0.8\linewidth]{imagenes/6_4_capacidad_umi_medio.png}
	\caption{Capacidad del receptor en la capa de micro-celda con una potencia de transmisión media.}
	\label{fig:simulacion_pot_med_umi}
\end{figure}

En este caso, el receptor recibe señal aceptable por parte de la capa de macro-celdas, viéndose reducido el tiempo sin conexión con respecto al anterior caso. Por otro lado, la capacidad de la capa de micro-celdas aumenta considerablemente, aumentando ligeramente el tiempo con suficiente cobertura.

Por último, se incrementa la potencia de transmisión a valores de 40 dBm para micro-celdas y 50 dBm para macro-celdas, lo cual resulta sustancialmente mayor de lo que se suele establecer en casos reales. Esta simulación tomó un tiempo de ejecución de 22 minutos y 37 segundos, y sus resultados pueden verse en las Figuras \ref{fig:simulacion_pot_max_uma} y \ref{fig:simulacion_pot_max_umi}.

\begin{figure}[h!]
	\centering
    \includegraphics[width=0.8\linewidth]{imagenes/6_4_capacidad_uma_max.png}
	\caption{Capacidad del receptor en la capa de macro-celda con una potencia de transmisión alta.}
	\label{fig:simulacion_pot_max_uma}
\end{figure}

\begin{figure}[h!]
	\centering
    \includegraphics[width=0.8\linewidth]{imagenes/6_4_capacidad_umi_max.png}
	\caption{Capacidad del receptor en la capa de micro-celda con una potencia de transmisión alta.}
	\label{fig:simulacion_pot_max_umi}
\end{figure}

Para el enlace con la macro-celda asociada, los tramos sin suficiente cobertura se ven ligeramente incrementados, a la vez que la capacidad media se ve reducida. Esta es una prueba directa de que la potencia de transmisión puede ser contraproducente en cuanto a interferencias se refiere.

Por otro lado, el enlace de micro-celdas se ve beneficiado con un incremento del tiempo de cobertura aceptable, disminuyendo zonas sin conexión, Sin embargo, este resultado no puede ser concluyente ni debe ser interpretado como una consecuencia directa de incrementar la potencia, puesto que el usuario se encuentra en el filo de las celdas de micro-celda, por lo que al aumentar la potencia, se mejoran las condiciones en esas situaciones, mientras que en situaciones más cercanas al centro de la celda, el rendimiento será menor debido a que las demás celdas interferirán con esa señal.

\subsection{Evolución de la separación geográfica}

Como última serie de experimentos, se ha diseñado un último conjunto de simulaciones en el que se estudia cómo afecta el tamaño de las celdas en el rendimiento de la red. Para ello, se realizarán tres simulaciones distintas donde se varía la distancia entre los centros de las celdas.

La simulación se ha establecido con unos parámetros estándar de 7 macro-celdas y 19 micro-celdas, con 40 dBm de potencia de transmisión y 2,6 GHz de frecuencia para macro-celdas y 27 dBm de potencia y 20 GHz de frecuencia para micro-celdas.

En la primera simulación se ha elegido una separación entre los centros de macro-celdas de 600 m, mientras que para los centros de micro-celdas se ha establecido una separación de 200 m. Esta simulación ha tomado 21 minutos y 8 segundos para su ejecución y pueden verse sus resultados en la Figura \ref{fig:simulacion_sep_min_uma} y en la Figura \ref{fig:simulacion_sep_min_umi}, donde está representada la capacidad del receptor para ambos casos.

\begin{figure}[h!]
	\centering
    \includegraphics[width=0.8\linewidth]{imagenes/6_5_capacidad_uma_min.png}
	\caption{Capacidad del receptor en la capa de macro-celda con un tamaño de celdas mínimo.}
	\label{fig:simulacion_sep_min_uma}
\end{figure}

\begin{figure}[h!]
	\centering
    \includegraphics[width=0.8\linewidth]{imagenes/6_5_capacidad_umi_min.png}
	\caption{Capacidad del receptor en la capa de micro-celda con un tamaño de celdas mínimo.}
	\label{fig:simulacion_sep_min_umi}
\end{figure}

En ambos resultados se aprecia un rendimiento pésimo por parte de la red, con valores máximos de capacidad de apenas 200 Mbit/s en macro-celdas, cuando la media en el resto de simulaciones había resultado en 500 Mbit/s. La zona de cobertura se ve reducida en comparación de igual modo.

En cuanto a micro-celdas, la cobertura, aun estando el receptor emparejado a la estación base de mayor SINR, es prácticamente nula, imposibilitando el servicio para dicho usuario.

Estos resultados son de esperar debido a la poca separación entre estaciones base que hacen que la interferencia se vea aumentada considerablemente.

Se repite la simulación para una separación de 5 km en macro-celdas y de 2,2 km para micro-celdas, con un tiempo de ejecución de 21 minutos y 7 segundos. Los resultados, que a priori se esperan parecidos a anteriores simulaciones, se representan en las Figuras \ref{fig:simulacion_sep_min_uma} y \ref{fig:simulacion_sep_min_uma}.

\begin{figure}[h!]
	\centering
    \includegraphics[width=0.8\linewidth]{imagenes/6_5_capacidad_uma_med.png}
	\caption{Capacidad del receptor en la capa de macro-celda con un tamaño de celdas medio.}
	\label{fig:simulacion_sep_med_uma}
\end{figure}

\begin{figure}[h!]
	\centering
    \includegraphics[width=0.8\linewidth]{imagenes/6_5_capacidad_umi_med.png}
	\caption{Capacidad del receptor en la capa de micro-celda con un tamaño de celdas medio.}
	\label{fig:simulacion_sep_med_umi}
\end{figure}

Como se puede observar, las condiciones para este tamaño de celdas se ven sustancialmente mejoradas, con una experiencia de servicio que resultaría satisfactoria para el usuario, ya que contaría con una cobertura estable en todo su recorrido para ambas capas de simulación.

La última simulación se realiza con una separación entre celdas de 11 km para macro-celdas, y 5 km para micro celdas. Esto supone que el escenario de simulación se extienda hasta aproximadamente 440 km$^{2}$, por lo que el kilómetro de la trayectoria del receptor es en proporción un recorrido corto. Los resultados pueden observarse en las Figuras \ref{fig:simulacion_sep_max_uma} y  \ref{fig:simulacion_sep_max_umi}, tomando un tiempo de simulación de 22 minutos y 14 segundos.

\begin{figure}[h!]
	\centering
    \includegraphics[width=0.8\linewidth]{imagenes/6_5_capacidad_uma_max.png}
	\caption{Capacidad del receptor en la capa de macro-celda con un tamaño de celdas máximo.}
	\label{fig:simulacion_sep_max_uma}
\end{figure}

\begin{figure}[ht!]
	\centering
    \includegraphics[width=0.8\linewidth]{imagenes/6_5_capacidad_umi_max.png}
	\caption{Capacidad del receptor en la capa de micro-celda con un tamaño de celdas máximo.}
	\label{fig:simulacion_sep_max_umi}
\end{figure}

En esta última ejecución se muestra un rendimiento de la red mayor que en el resto de simulaciones anteriores, con un tiempo de servicio y de cobertura estable de aproximadamente el 100\% en el caso de macro-celda, y muy cercano al total también en micro-celda.

Esto demuestra que el principal problema que afecta al rendimiento de la red son las interferencias, puesto que al alejar la fuente de interferencias del receptor, la capacidad asignada al mismo se ve considerablemente mejorada. En este caso, las mejores condiciones para el escenario son las de aumentar el tamaño de celda para asegurar una mayor cobertura en las mismas sin interferencias.

Los resultados refuerzan la necesidad de incorporar al simulador algún sistema de planificación que implemente reuso de frecuencias, con la finalidad de mejorar el rendimiento de la red y de asemejarse a los sistemas que se utilizan en la actualidad.
%
\chapter{Conclusiones y trabajos futuros}\label{cap.conclusiones}
\section{Introducción}
Bla bla bla
%
%\chapter{Conclusiones y Trabajos Futuros}
%
%
%%\nocite{*}
%\addbibresource{bibliografia/bibliografia.bib}
\bibliography{./bibliografia/bibliografia}
\bibliographystyle{unsrt}
%
%\appendix
%\input{apendices/manual_usuario/manual_usuario}
%%\input{apendices/paper/paper}
%\input{glosario/entradas_glosario}
% \addcontentsline{toc}{chapter}{Glosario}
% \printglossary
\chapter*{}
\thispagestyle{empty}

\end{document}
