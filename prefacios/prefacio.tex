%\chapter*{}
%\thispagestyle{empty}
%\cleardoublepage

%\thispagestyle{empty}

\input{portada/portada_2}

\cleardoublepage
\thispagestyle{empty}

\begin{center}
{\large\bfseries \myTitle: Integración del simulador de canal de propagación \textit{QuaDRiGa} en un simulador de nivel de sistema}\\
\end{center}
\begin{center}
\myName\\
\end{center}

%\vspace{0.7cm}
\noindent{\textbf{Palabras clave}: 5G, Simulador, Redes 5G, HetNets, Comunicaciones de Radio, Comunicaciones Inalámbricas, IoT, Macro-Celdas, Micro-Celdas, QuaDRiGa, Capacidad de Canal, Simulador de Nivel de Sistema }\\

\vspace{0.7cm}
\noindent{\textbf{Resumen}}\\

La evolución de las tecnologías inalámbricas está marcada por el gran crecimiento de usuarios y de terminales móviles que hacen uso de las infraestructuras de comunicación. Tal es el cambio que las telecomunicaciones inalámbricas están sufriendo que se espera que en los próximos tres años, el número de usuarios de las infraestructuras se vea incrementado en más de un 15\% a nivel mundial, con un volumen de tráfico de red tres veces mayor que el actual.

Con la finalidad de satisfacer los rigurosos requisitos de las nuevas tecnologías de radiocomunicación y facilitar su planificación y despliegue, han sido impulsadas ciertas iniciativas por parte del sector industrial y el sector académico. Una de ellas es \textit{QuaDRiGa}, una herramienta de simulación que ha sido utilizada por diversos proyectos europeos para generar respuestas de canales de radio realistas que son compatibles con los modelos estandarizados por el 3rd Generation Partnership Project, 3GPP. 

Este proyecto amplía las funcionalidades de \textit{QuaDRiGa} para producir simulaciones a nivel de red de escenarios 5G, con la adición de cierta cantidad de mejoras que hacen que las simulaciones otorguen resultados significativos, de una forma eficiente en cuanto a los aspectos computacionales se refiere. Este simulador, denominando \textit{5Gneralife} ha sido desarrollado en Matlab con el ánimo de que resulte fácil de usar y que contribuya a los fines académicos.

Además, se realiza una serie de evaluaciones del simulador en variedad de escenarios, desde escenarios realistas para estudiar el rendimiento de la red, hasta modificando parámetros con el propósito de determinar cómo afectan las configuraciones de la infraestructura al desempeño de la misma en términos de capacidad y cobertura.
\cleardoublepage


\thispagestyle{empty}


\begin{center}
{\large\bfseries \textit{5Gneralife}: Mobile communications simulator for 5G networks}\\
\end{center}
\begin{center}
Francisco Quero de la Rosa\\
\end{center}

%\vspace{0.7cm}
\noindent{\textbf{Keywords}: 5G, Simulator, HetNets, Radio-Communications, Wireless Communications, IoT, Large Cells, Small Cells, QuaDRiGa, Channel Capacity, Network Level Simulator}\\

\vspace{0.7cm}
\noindent{\textbf{Abstract}}\\

Wireless technologies evolution is influenced by the strong user and mobile terminals growth. There is such a change of wireless telecommunication paradigm that users amount around world is being expected to grow by 15 percent in the next three years, with a traffic volume boost from 17 to 50 exabytes.

With the aim of satisfying the stringent requirements of the new emerging technologies and facilitating their planning and deployment, numerous initiatives have been taken from the industry and academia sectors. One of them is \textit{QuaDRiGa}, a simulation-tool currently used in several European projects to generate realistic radio channel impulse responses which are compatible with \textit{3rd Generation Partnership Project, 3GPP} standardized channel models. 

This project extends \textit{QuaDRiGa} to produce network-level simulations of relevant 5G scenarios in a computationally-efficient manner, which includes several improvements which contribute to achieve a more versatile simulator. This 5G simulator, called \textit{5Gneralife}, has been developed in Matlab with the aim of being intuitive in order to hope to be useful to the academic community.

In addition, performance evaluations in terms of capacity and coverage are provided for several realistic use cases, as well as some experimentation simulations that have as purpose to determine how the enviroment is affected by parameter variation.

\chapter*{}
\thispagestyle{empty}

\noindent\rule[-1ex]{\textwidth}{2pt}\\[4.5ex]

Yo, \textbf{\myName}, alumno de la titulación \myDegree de la \textbf{Escuela Técnica Superior
de Ingenierías Informática y de Telecomunicación de la Universidad de Granada}, con DNI 77382810-T, autorizo la
ubicación de la siguiente copia de mi Trabajo Fin de Grado en la biblioteca del centro para que pueda ser
consultada por las personas que lo deseen.

\vspace{6cm}

\noindent Fdo: \myName

\vspace{2cm}

\begin{flushright}
Granada a \myTime.
\end{flushright}


\chapter*{}
\thispagestyle{empty}

\noindent\rule[-1ex]{\textwidth}{2pt}\\[4.5ex]

D. \textbf{\myProf}, Profesor del Área de Telemática del Departamento \myDepartment de la Universidad de Granada.

\vspace{0.5cm}

D. \textbf{\myOtherProf}, Profesor del Área de Telemática del Departamento \myDepartment de la Universidad de Granada.


\vspace{0.5cm}

\textbf{Informan:}

\vspace{0.5cm}

Que el presente trabajo, titulado \textit{\textbf{\myTitle}},
ha sido realizado bajo su supervisión por \textbf{\myName}, y autorizamos la defensa de dicho trabajo ante el tribunal
que corresponda.

\vspace{0.5cm}

Y para que conste, expiden y firman el presente informe en Granada a \myTime.

\vspace{1cm}

\textbf{Los directores:}

\vspace{5cm}

\noindent \textbf{\myProf} \ \ \ \ \ \textbf{\myOtherProf}

\chapter*{Agradecimientos}
\thispagestyle{empty}

       \vspace{1cm}


Quiero dedicar unas palabras a todas las personas que han formado parte de mis últimos cuatro años de andanzas, sin ellos, todo habría sido más cuesta arriba.

En especial, a mis padres, Francisco y Ana Mari,los mejores que se pueden tener, que me han dado todas las facilidades del mundo para elegir la senda que yo quería, sin importar nada más que ofrecerme lo mejor para mí. Y aún a día de hoy, con todo acabado, siguen ofreciéndomelo. Y a mi familia que tantos ratos de charla sobre cómo me iba, y que tantas palabras de ánimo me han regalado.

Cómo no, también a Ana por todo su cariño y compañía en los días que más lo he necesitado (y por soportarme), por haber querido compartir conmigo el final de esta etapa, y por querer compartir el comienzo de la siguiente.

A mis tutores, Pablo y Juanfra por su total disponibilidad siempre que lo he necesitado, y también por su paciencia y magníficas labores docentes y de tutelaje no solo académico, sino también personal y profesional. 

A ese porcentaje reducido de profesores y profesoras que transmiten más que conocimientos, que han despertado en mí inquietudes que no existían y que se pueden considerar verdaderos docentes.

A mis compañeros del Grado, que tantas horas hemos pasado juntos, en especial a Carmelo, Sergio, David, Adri, Gabri y Gásquez, cuya complicidad demuestra que el grado me ha regalado más que compañeros, me ha regalado amigos inseparables.

A mis amigos del CITIC, Jose Manuel, Pilar, José Antonio, Óscar, Ángel y Pablo, con los que he compartido mis últimos seis meses del grado entre risas, café y trabajo duro.

A todos mis amigos de Porcuna que, por haber elegido Granada como lugar de estudio, con todo mi pesar he tenido que dejar de ver tanto, y que aun así, siempre están cuando se les necesita.

Y en definitiva, a toda persona que haya vivido conmigo parte de esta carrera de fondo que no acaba más que empezar. Aunque me haya costado algún sacrificio, hoy, más que nunca, me alegro de haber tomado este camino. Sé que me he extendido pero todo agradecimiento se queda corto para aquellos que lo merecen. Gracias.